% Options for packages loaded elsewhere
% Options for packages loaded elsewhere
\PassOptionsToPackage{unicode}{hyperref}
\PassOptionsToPackage{hyphens}{url}
\PassOptionsToPackage{dvipsnames,svgnames,x11names}{xcolor}
%
\documentclass[
  letterpaper,
  DIV=11,
  numbers=noendperiod]{scrartcl}
\usepackage{xcolor}
\usepackage{amsmath,amssymb}
\setcounter{secnumdepth}{5}
\usepackage{iftex}
\ifPDFTeX
  \usepackage[T1]{fontenc}
  \usepackage[utf8]{inputenc}
  \usepackage{textcomp} % provide euro and other symbols
\else % if luatex or xetex
  \usepackage{unicode-math} % this also loads fontspec
  \defaultfontfeatures{Scale=MatchLowercase}
  \defaultfontfeatures[\rmfamily]{Ligatures=TeX,Scale=1}
\fi
\usepackage{lmodern}
\ifPDFTeX\else
  % xetex/luatex font selection
\fi
% Use upquote if available, for straight quotes in verbatim environments
\IfFileExists{upquote.sty}{\usepackage{upquote}}{}
\IfFileExists{microtype.sty}{% use microtype if available
  \usepackage[]{microtype}
  \UseMicrotypeSet[protrusion]{basicmath} % disable protrusion for tt fonts
}{}
\makeatletter
\@ifundefined{KOMAClassName}{% if non-KOMA class
  \IfFileExists{parskip.sty}{%
    \usepackage{parskip}
  }{% else
    \setlength{\parindent}{0pt}
    \setlength{\parskip}{6pt plus 2pt minus 1pt}}
}{% if KOMA class
  \KOMAoptions{parskip=half}}
\makeatother
% Make \paragraph and \subparagraph free-standing
\makeatletter
\ifx\paragraph\undefined\else
  \let\oldparagraph\paragraph
  \renewcommand{\paragraph}{
    \@ifstar
      \xxxParagraphStar
      \xxxParagraphNoStar
  }
  \newcommand{\xxxParagraphStar}[1]{\oldparagraph*{#1}\mbox{}}
  \newcommand{\xxxParagraphNoStar}[1]{\oldparagraph{#1}\mbox{}}
\fi
\ifx\subparagraph\undefined\else
  \let\oldsubparagraph\subparagraph
  \renewcommand{\subparagraph}{
    \@ifstar
      \xxxSubParagraphStar
      \xxxSubParagraphNoStar
  }
  \newcommand{\xxxSubParagraphStar}[1]{\oldsubparagraph*{#1}\mbox{}}
  \newcommand{\xxxSubParagraphNoStar}[1]{\oldsubparagraph{#1}\mbox{}}
\fi
\makeatother

\usepackage{color}
\usepackage{fancyvrb}
\newcommand{\VerbBar}{|}
\newcommand{\VERB}{\Verb[commandchars=\\\{\}]}
\DefineVerbatimEnvironment{Highlighting}{Verbatim}{commandchars=\\\{\}}
% Add ',fontsize=\small' for more characters per line
\usepackage{framed}
\definecolor{shadecolor}{RGB}{241,243,245}
\newenvironment{Shaded}{\begin{snugshade}}{\end{snugshade}}
\newcommand{\AlertTok}[1]{\textcolor[rgb]{0.68,0.00,0.00}{#1}}
\newcommand{\AnnotationTok}[1]{\textcolor[rgb]{0.37,0.37,0.37}{#1}}
\newcommand{\AttributeTok}[1]{\textcolor[rgb]{0.40,0.45,0.13}{#1}}
\newcommand{\BaseNTok}[1]{\textcolor[rgb]{0.68,0.00,0.00}{#1}}
\newcommand{\BuiltInTok}[1]{\textcolor[rgb]{0.00,0.23,0.31}{#1}}
\newcommand{\CharTok}[1]{\textcolor[rgb]{0.13,0.47,0.30}{#1}}
\newcommand{\CommentTok}[1]{\textcolor[rgb]{0.37,0.37,0.37}{#1}}
\newcommand{\CommentVarTok}[1]{\textcolor[rgb]{0.37,0.37,0.37}{\textit{#1}}}
\newcommand{\ConstantTok}[1]{\textcolor[rgb]{0.56,0.35,0.01}{#1}}
\newcommand{\ControlFlowTok}[1]{\textcolor[rgb]{0.00,0.23,0.31}{\textbf{#1}}}
\newcommand{\DataTypeTok}[1]{\textcolor[rgb]{0.68,0.00,0.00}{#1}}
\newcommand{\DecValTok}[1]{\textcolor[rgb]{0.68,0.00,0.00}{#1}}
\newcommand{\DocumentationTok}[1]{\textcolor[rgb]{0.37,0.37,0.37}{\textit{#1}}}
\newcommand{\ErrorTok}[1]{\textcolor[rgb]{0.68,0.00,0.00}{#1}}
\newcommand{\ExtensionTok}[1]{\textcolor[rgb]{0.00,0.23,0.31}{#1}}
\newcommand{\FloatTok}[1]{\textcolor[rgb]{0.68,0.00,0.00}{#1}}
\newcommand{\FunctionTok}[1]{\textcolor[rgb]{0.28,0.35,0.67}{#1}}
\newcommand{\ImportTok}[1]{\textcolor[rgb]{0.00,0.46,0.62}{#1}}
\newcommand{\InformationTok}[1]{\textcolor[rgb]{0.37,0.37,0.37}{#1}}
\newcommand{\KeywordTok}[1]{\textcolor[rgb]{0.00,0.23,0.31}{\textbf{#1}}}
\newcommand{\NormalTok}[1]{\textcolor[rgb]{0.00,0.23,0.31}{#1}}
\newcommand{\OperatorTok}[1]{\textcolor[rgb]{0.37,0.37,0.37}{#1}}
\newcommand{\OtherTok}[1]{\textcolor[rgb]{0.00,0.23,0.31}{#1}}
\newcommand{\PreprocessorTok}[1]{\textcolor[rgb]{0.68,0.00,0.00}{#1}}
\newcommand{\RegionMarkerTok}[1]{\textcolor[rgb]{0.00,0.23,0.31}{#1}}
\newcommand{\SpecialCharTok}[1]{\textcolor[rgb]{0.37,0.37,0.37}{#1}}
\newcommand{\SpecialStringTok}[1]{\textcolor[rgb]{0.13,0.47,0.30}{#1}}
\newcommand{\StringTok}[1]{\textcolor[rgb]{0.13,0.47,0.30}{#1}}
\newcommand{\VariableTok}[1]{\textcolor[rgb]{0.07,0.07,0.07}{#1}}
\newcommand{\VerbatimStringTok}[1]{\textcolor[rgb]{0.13,0.47,0.30}{#1}}
\newcommand{\WarningTok}[1]{\textcolor[rgb]{0.37,0.37,0.37}{\textit{#1}}}

\usepackage{longtable,booktabs,array}
\usepackage{calc} % for calculating minipage widths
% Correct order of tables after \paragraph or \subparagraph
\usepackage{etoolbox}
\makeatletter
\patchcmd\longtable{\par}{\if@noskipsec\mbox{}\fi\par}{}{}
\makeatother
% Allow footnotes in longtable head/foot
\IfFileExists{footnotehyper.sty}{\usepackage{footnotehyper}}{\usepackage{footnote}}
\makesavenoteenv{longtable}
\usepackage{graphicx}
\makeatletter
\newsavebox\pandoc@box
\newcommand*\pandocbounded[1]{% scales image to fit in text height/width
  \sbox\pandoc@box{#1}%
  \Gscale@div\@tempa{\textheight}{\dimexpr\ht\pandoc@box+\dp\pandoc@box\relax}%
  \Gscale@div\@tempb{\linewidth}{\wd\pandoc@box}%
  \ifdim\@tempb\p@<\@tempa\p@\let\@tempa\@tempb\fi% select the smaller of both
  \ifdim\@tempa\p@<\p@\scalebox{\@tempa}{\usebox\pandoc@box}%
  \else\usebox{\pandoc@box}%
  \fi%
}
% Set default figure placement to htbp
\def\fps@figure{htbp}
\makeatother





\setlength{\emergencystretch}{3em} % prevent overfull lines

\providecommand{\tightlist}{%
  \setlength{\itemsep}{0pt}\setlength{\parskip}{0pt}}



 


% load packages
\usepackage{geometry}
\usepackage{xcolor}
\usepackage{eso-pic}
\usepackage{fancyhdr}
\usepackage{sectsty}
\usepackage{fontspec}
\usepackage{titlesec}

%% Set page size with a wider right margin
\geometry{a4paper, total={170mm,257mm}, left=20mm, top=20mm, bottom=20mm, right=50mm}

%% Let's define some colours
\definecolor{light}{HTML}{E6E6FA}
\definecolor{highlight}{HTML}{800080}
\definecolor{dark}{HTML}{330033}

%% Let's add the border on the right hand side 
\AddToShipoutPicture{% 
    \AtPageLowerLeft{% 
        \put(\LenToUnit{\dimexpr\paperwidth-3cm},0){% 
            \color{light}\rule{3cm}{\LenToUnit\paperheight}%
          }%
     }%
     % logo
    \AtPageLowerLeft{% start the bar at the bottom right of the page
        \put(\LenToUnit{\dimexpr\paperwidth-2.25cm},27.2cm){% move it to the top right
            \color{light}\includegraphics[width=1.5cm]{_extensions/nrennie/PrettyPDF/logo.png}
          }%
     }%
}

%% Style the page number
\fancypagestyle{mystyle}{
  \fancyhf{}
  \renewcommand\headrulewidth{0pt}
  \fancyfoot[R]{\thepage}
  \fancyfootoffset{3.5cm}
}
\setlength{\footskip}{20pt}

%% style the chapter/section fonts
\chapterfont{\color{dark}\fontsize{20}{16.8}\selectfont}
\sectionfont{\color{dark}\fontsize{20}{16.8}\selectfont}
\subsectionfont{\color{dark}\fontsize{14}{16.8}\selectfont}
\titleformat{\subsection}
  {\sffamily\Large\bfseries}{\thesection}{1em}{}[{\titlerule[0.8pt]}]
  
% left align title
\makeatletter
\renewcommand{\maketitle}{\bgroup\setlength{\parindent}{0pt}
\begin{flushleft}
  {\sffamily\huge\textbf{\MakeUppercase{\@title}}} \vspace{0.3cm} \newline
  {\Large {\@subtitle}} \newline
  \@author
\end{flushleft}\egroup
}
\makeatother

%% Use some custom fonts
\setsansfont{Ubuntu}[
    Path=_extensions/nrennie/PrettyPDF/Ubuntu/,
    Scale=0.9,
    Extension = .ttf,
    UprightFont=*-Regular,
    BoldFont=*-Bold,
    ItalicFont=*-Italic,
    ]

\setmainfont{Ubuntu}[
    Path=_extensions/nrennie/PrettyPDF/Ubuntu/,
    Scale=0.9,
    Extension = .ttf,
    UprightFont=*-Regular,
    BoldFont=*-Bold,
    ItalicFont=*-Italic,
    ]
\KOMAoption{captions}{tableheading}
\makeatletter
\@ifpackageloaded{tcolorbox}{}{\usepackage[skins,breakable]{tcolorbox}}
\@ifpackageloaded{fontawesome5}{}{\usepackage{fontawesome5}}
\definecolor{quarto-callout-color}{HTML}{909090}
\definecolor{quarto-callout-note-color}{HTML}{0758E5}
\definecolor{quarto-callout-important-color}{HTML}{CC1914}
\definecolor{quarto-callout-warning-color}{HTML}{EB9113}
\definecolor{quarto-callout-tip-color}{HTML}{00A047}
\definecolor{quarto-callout-caution-color}{HTML}{FC5300}
\definecolor{quarto-callout-color-frame}{HTML}{acacac}
\definecolor{quarto-callout-note-color-frame}{HTML}{4582ec}
\definecolor{quarto-callout-important-color-frame}{HTML}{d9534f}
\definecolor{quarto-callout-warning-color-frame}{HTML}{f0ad4e}
\definecolor{quarto-callout-tip-color-frame}{HTML}{02b875}
\definecolor{quarto-callout-caution-color-frame}{HTML}{fd7e14}
\makeatother
\makeatletter
\@ifpackageloaded{caption}{}{\usepackage{caption}}
\AtBeginDocument{%
\ifdefined\contentsname
  \renewcommand*\contentsname{Table of contents}
\else
  \newcommand\contentsname{Table of contents}
\fi
\ifdefined\listfigurename
  \renewcommand*\listfigurename{List of Figures}
\else
  \newcommand\listfigurename{List of Figures}
\fi
\ifdefined\listtablename
  \renewcommand*\listtablename{List of Tables}
\else
  \newcommand\listtablename{List of Tables}
\fi
\ifdefined\figurename
  \renewcommand*\figurename{Figure}
\else
  \newcommand\figurename{Figure}
\fi
\ifdefined\tablename
  \renewcommand*\tablename{Table}
\else
  \newcommand\tablename{Table}
\fi
}
\@ifpackageloaded{float}{}{\usepackage{float}}
\floatstyle{ruled}
\@ifundefined{c@chapter}{\newfloat{codelisting}{h}{lop}}{\newfloat{codelisting}{h}{lop}[chapter]}
\floatname{codelisting}{Listing}
\newcommand*\listoflistings{\listof{codelisting}{List of Listings}}
\makeatother
\makeatletter
\makeatother
\makeatletter
\@ifpackageloaded{caption}{}{\usepackage{caption}}
\@ifpackageloaded{subcaption}{}{\usepackage{subcaption}}
\makeatother
\makeatletter
\@ifpackageloaded{tcolorbox}{}{\usepackage[skins,breakable]{tcolorbox}}
\makeatother
\makeatletter
\@ifundefined{shadecolor}{\definecolor{shadecolor}{rgb}{.97, .97, .97}}{}
\makeatother
\makeatletter
\@ifundefined{codebgcolor}{\definecolor{codebgcolor}{named}{light}}{}
\makeatother
\makeatletter
\ifdefined\Shaded\renewenvironment{Shaded}{\begin{tcolorbox}[colback={codebgcolor}, sharp corners, boxrule=0pt, breakable, enhanced, frame hidden]}{\end{tcolorbox}}\fi
\makeatother
\usepackage{bookmark}
\IfFileExists{xurl.sty}{\usepackage{xurl}}{} % add URL line breaks if available
\urlstyle{same}
\hypersetup{
  pdftitle={Practical 6},
  colorlinks=true,
  linkcolor={highlight},
  filecolor={Maroon},
  citecolor={Blue},
  urlcolor={highlight},
  pdfcreator={LaTeX via pandoc}}


\title{Practical 6}
\author{}
\date{}
\begin{document}
\maketitle

\pagestyle{mystyle}


\textbf{Aim of this practical:}

In this practical we are going to look at space-time model and model
comparison and validation techniques.

\subsection{Space time models}\label{space-time-models}

\begin{center}\rule{0.5\linewidth}{0.5pt}\end{center}

Libraries to load:

\begin{Shaded}
\begin{Highlighting}[]
\FunctionTok{library}\NormalTok{(dplyr)}
\FunctionTok{library}\NormalTok{(INLA)}
\FunctionTok{library}\NormalTok{(inlabru) }
\FunctionTok{library}\NormalTok{(sf)}
\FunctionTok{library}\NormalTok{(terra)}


\CommentTok{\# load some libraries to generate nice map plots}
\FunctionTok{library}\NormalTok{(scico)}
\FunctionTok{library}\NormalTok{(ggplot2)}
\FunctionTok{library}\NormalTok{(patchwork)}
\FunctionTok{library}\NormalTok{(mapview)}
\FunctionTok{library}\NormalTok{(tidyterra)}
\end{Highlighting}
\end{Shaded}

\subsubsection{The data}\label{the-data}

In this practical, we will revisit the data on the Pacific Cod
(\emph{Gadus macrocephalus}) from a trawl survey in Queen Charlotte
Sound. The \texttt{pcod} dataset is available from the \texttt{sdmTMB}
package and contains the presence/absence records of the Pacific Cod
during each surveys along with the biomass density of Pacific cod in the
area swept (kg/Km\(^2\)). The \texttt{qcs\_grid} data contain the depth
values stored as \(2\times 2\) km grid for Queen Charlotte Sound.

The dataset contains presence/absence data from 2003 to 2017. To make
computations faster we only consider the first 3 years.

\begin{Shaded}
\begin{Highlighting}[]
\FunctionTok{library}\NormalTok{(sdmTMB)}

\NormalTok{pcod\_df }\OtherTok{=}\NormalTok{ sdmTMB}\SpecialCharTok{::}\NormalTok{pcod  }\SpecialCharTok{\%\textgreater{}\%} \FunctionTok{filter}\NormalTok{(year}\SpecialCharTok{\textless{}=}\DecValTok{2005}\NormalTok{)}
\NormalTok{qcs\_grid }\OtherTok{=}\NormalTok{ sdmTMB}\SpecialCharTok{::}\NormalTok{qcs\_grid}
\end{Highlighting}
\end{Shaded}

Then, we create a \texttt{sf} object and assign the right coordinate
reference to it:

\begin{Shaded}
\begin{Highlighting}[]
\NormalTok{pcod\_sf }\OtherTok{=}   \FunctionTok{st\_as\_sf}\NormalTok{(pcod\_df, }\AttributeTok{coords =} \FunctionTok{c}\NormalTok{(}\StringTok{"lon"}\NormalTok{,}\StringTok{"lat"}\NormalTok{), }\AttributeTok{crs =} \DecValTok{4326}\NormalTok{)}
\NormalTok{pcod\_sf }\OtherTok{=} \FunctionTok{st\_transform}\NormalTok{(pcod\_sf,}
          \AttributeTok{crs =} \StringTok{"+proj=utm +zone=9 +datum=WGS84 +no\_defs +type=crs +units=km"}\NormalTok{ )}
\end{Highlighting}
\end{Shaded}

We convert the covariates into a raster and assign the same coordinate
reference:

\begin{Shaded}
\begin{Highlighting}[]
\NormalTok{depth\_r }\OtherTok{\textless{}{-}} \FunctionTok{rast}\NormalTok{(qcs\_grid, }\AttributeTok{type =} \StringTok{"xyz"}\NormalTok{)}
\FunctionTok{crs}\NormalTok{(depth\_r) }\OtherTok{\textless{}{-}} \FunctionTok{crs}\NormalTok{(pcod\_sf)}
\end{Highlighting}
\end{Shaded}

\begin{center}
\pandocbounded{\includegraphics[keepaspectratio]{day4_practical_6_files/figure-pdf/unnamed-chunk-7-1.pdf}}
\end{center}

\subsubsection{Spatio-temporal modeling}\label{spatio-temporal-modeling}

\paragraph{Model fitting}\label{model-fitting}

Now lets compare two different space-time models using LGOCV and some
information criteria metrics. The general model structure is given by:

\[
\begin{aligned}
y(s,t)|\eta(s,t)&\sim\text{Binom}(1, p(s,t))\\
\eta(s,t) &= \text{logit}(p(s,t)) \\
\end{aligned}
\] We also want to compare the models using WAIC, DIC and marginal
likelihood:

\begin{tcolorbox}[enhanced jigsaw, coltitle=black, rightrule=.15mm, bottomrule=.15mm, left=2mm, arc=.35mm, bottomtitle=1mm, colbacktitle=quarto-callout-warning-color!10!white, leftrule=.75mm, title={Task}, toprule=.15mm, opacitybacktitle=0.6, breakable, colback=white, colframe=quarto-callout-warning-color-frame, toptitle=1mm, opacityback=0, titlerule=0mm]

Set the \texttt{bru\_options} so that the quantities of interest are
computed

Click here to see the solution

\begin{Shaded}
\begin{Highlighting}[]
\FunctionTok{bru\_options\_set}\NormalTok{(}\AttributeTok{control.compute =} \FunctionTok{list}\NormalTok{(}\AttributeTok{waic =} \ConstantTok{TRUE}\NormalTok{,}\AttributeTok{dic=} \ConstantTok{TRUE}\NormalTok{,}\AttributeTok{mlik =} \ConstantTok{TRUE}\NormalTok{))}
\end{Highlighting}
\end{Shaded}

\end{tcolorbox}

\paragraph{Model 1}\label{model-1}

\begin{enumerate}
\def\labelenumi{\arabic{enumi}.}
\tightlist
\item
  \textbf{Model 1 - time iid effect} We consider a separable space-time
  model with a linear predictors given by:
\end{enumerate}

\[
\eta(s,t) = \beta_0 + f_1(\text{depth}(s)) + f_2(t) + \omega(s)
\]

\begin{itemize}
\item
  \(f_1(\text{depth}(s))\) is a smooth covariate effect of depth
  (modeled using a RW2 model)
\item
  \(f_2(t)\) is an IID effect of time
\item
  \(\omega(s)\) is Matérn random field.
\end{itemize}

The first step is to define the mesh and the spde model

\textbf{Construct the mesh and the SPDE model}

\begin{Shaded}
\begin{Highlighting}[]
\NormalTok{mesh }\OtherTok{=} \FunctionTok{fm\_mesh\_2d}\NormalTok{(}\AttributeTok{loc =}\NormalTok{ pcod\_sf,    }
                  \AttributeTok{cutoff =} \DecValTok{1}\NormalTok{,}
                  \AttributeTok{max.edge =} \FunctionTok{c}\NormalTok{(}\DecValTok{10}\NormalTok{,}\DecValTok{20}\NormalTok{),     }
                  \AttributeTok{offset =} \FunctionTok{c}\NormalTok{(}\DecValTok{5}\NormalTok{,}\DecValTok{50}\NormalTok{),}
                  \AttributeTok{crs =} \FunctionTok{st\_crs}\NormalTok{(pcod\_df))   }

\NormalTok{spde\_model }\OtherTok{=}  \FunctionTok{inla.spde2.pcmatern}\NormalTok{(mesh,}
                                  \AttributeTok{prior.sigma =} \FunctionTok{c}\NormalTok{(}\DecValTok{1}\NormalTok{, }\FloatTok{0.5}\NormalTok{),}
                                  \AttributeTok{prior.range =} \FunctionTok{c}\NormalTok{(}\DecValTok{100}\NormalTok{, }\FloatTok{0.5}\NormalTok{))}
\end{Highlighting}
\end{Shaded}

\begin{center}
\pandocbounded{\includegraphics[keepaspectratio]{day4_practical_6_files/figure-pdf/unnamed-chunk-10-1.pdf}}
\end{center}

\textbf{create time index and the grouped variable}

To use the RW2 model the covariate has to be groupes:

\begin{Shaded}
\begin{Highlighting}[]
\NormalTok{depth\_r}\SpecialCharTok{$}\NormalTok{depth\_group }\OtherTok{=} \FunctionTok{inla.group}\NormalTok{(}\FunctionTok{values}\NormalTok{(depth\_r}\SpecialCharTok{$}\NormalTok{depth\_scaled))}
\end{Highlighting}
\end{Shaded}

we also define a time index from 1 to in the data frame

\begin{Shaded}
\begin{Highlighting}[]
\NormalTok{pcod\_sf }\OtherTok{=}\NormalTok{ pcod\_sf }\SpecialCharTok{\%\textgreater{}\%}
     \FunctionTok{mutate}\NormalTok{(}\AttributeTok{time\_idx =} \FunctionTok{match}\NormalTok{(year, }\FunctionTok{c}\NormalTok{(}\DecValTok{2003}\NormalTok{, }\DecValTok{2004}\NormalTok{, }\DecValTok{2005}\NormalTok{)),}
         \AttributeTok{id =} \DecValTok{1}\SpecialCharTok{:}\FunctionTok{nrow}\NormalTok{(.)) }\CommentTok{\# Observation id for CV}
\end{Highlighting}
\end{Shaded}

\begin{tcolorbox}[enhanced jigsaw, coltitle=black, rightrule=.15mm, bottomrule=.15mm, left=2mm, arc=.35mm, bottomtitle=1mm, colbacktitle=quarto-callout-warning-color!10!white, leftrule=.75mm, title={Task}, toprule=.15mm, opacitybacktitle=0.6, breakable, colback=white, colframe=quarto-callout-warning-color-frame, toptitle=1mm, opacityback=0, titlerule=0mm]

Implement the model in \texttt{inlabru}

\begin{enumerate}
\def\labelenumi{\arabic{enumi}.}
\tightlist
\item
  Define the components
\item
  Define the formula
\item
  Define the likelihood model using the \texttt{bru\_obs()} function
\item
  Run the model
\end{enumerate}

Click here to see the solution

\begin{Shaded}
\begin{Highlighting}[]
\CommentTok{\# Model components}
\NormalTok{cmp\_spat }\OtherTok{=} \ErrorTok{\textasciitilde{}} \FunctionTok{Intercept}\NormalTok{(}\DecValTok{1}\NormalTok{) }\SpecialCharTok{+} 
  \FunctionTok{covariate}\NormalTok{(depth\_r}\SpecialCharTok{$}\NormalTok{depth\_group, }\AttributeTok{model =} \StringTok{"rw2"}\NormalTok{, }\AttributeTok{scale.model =} \ConstantTok{TRUE}\NormalTok{)}\SpecialCharTok{+}
  \FunctionTok{trend}\NormalTok{(time\_idx, }\AttributeTok{model =} \StringTok{"iid"}\NormalTok{)}\SpecialCharTok{+}
  \FunctionTok{space}\NormalTok{(geometry, }\AttributeTok{model =}\NormalTok{ spde\_model)}

\CommentTok{\# Linear predictor}
\NormalTok{formula\_spat }\OtherTok{=}\NormalTok{ present }\SpecialCharTok{\textasciitilde{}}\NormalTok{ Intercept  }\SpecialCharTok{+}\NormalTok{ trend }\SpecialCharTok{+}\NormalTok{ space }\SpecialCharTok{+}\NormalTok{ covariate}

\CommentTok{\# Observational model}
\NormalTok{lik\_spat }\OtherTok{=} \FunctionTok{bru\_obs}\NormalTok{(}\AttributeTok{formula =}\NormalTok{ formula\_spat, }
              \AttributeTok{data =}\NormalTok{ pcod\_sf, }
              \AttributeTok{family =} \StringTok{"binomial"}\NormalTok{)}

\CommentTok{\# Fit Model }
\NormalTok{fit\_spat }\OtherTok{=} \FunctionTok{bru}\NormalTok{(cmp\_spat,lik\_spat)}
\end{Highlighting}
\end{Shaded}

\end{tcolorbox}

\begin{tcolorbox}[enhanced jigsaw, coltitle=black, rightrule=.15mm, bottomrule=.15mm, left=2mm, arc=.35mm, bottomtitle=1mm, colbacktitle=quarto-callout-note-color!10!white, leftrule=.75mm, title=\textcolor{quarto-callout-note-color}{\faInfo}\hspace{0.5em}{Note}, toprule=.15mm, opacitybacktitle=0.6, breakable, colback=white, colframe=quarto-callout-note-color-frame, toptitle=1mm, opacityback=0, titlerule=0mm]

Note that there are some survey locations in certain years that fall
outside the depth raster region. \texttt{inlabru} will input these
missing covariate values using the nearest available value. This can be
computationally expensive, but you can avoid it by supplying a raster
layer that encompasses all of your data points (e.g., by pre-imputing
these missing values with your preferred method of choice).

One way of doing this in the \texttt{inlabru} framework is to use the
\texttt{bru\_fill\_missing()} function:

\begin{Shaded}
\begin{Highlighting}[]
\CommentTok{\# Select the raster of interest}
\NormalTok{depth\_orig }\OtherTok{=}\NormalTok{ depth\_r}\SpecialCharTok{$}\NormalTok{depth\_group}
\NormalTok{re }\OtherTok{\textless{}{-}} \FunctionTok{extend}\NormalTok{(depth\_orig, }\FunctionTok{ext}\NormalTok{(depth\_orig)}\SpecialCharTok{*}\FloatTok{1.05}\NormalTok{)}
\CommentTok{\# Convert to an sf spatial object}
\NormalTok{re\_df }\OtherTok{\textless{}{-}}\NormalTok{ re }\SpecialCharTok{\%\textgreater{}\%}\NormalTok{ stars}\SpecialCharTok{::}\FunctionTok{st\_as\_stars}\NormalTok{() }\SpecialCharTok{\%\textgreater{}\%}  \FunctionTok{st\_as\_sf}\NormalTok{(}\AttributeTok{na.rm=}\NormalTok{F)}
\CommentTok{\# fill in missing values using the original raster }
\NormalTok{re\_df}\SpecialCharTok{$}\NormalTok{depth\_group }\OtherTok{=}  \FunctionTok{bru\_fill\_missing}\NormalTok{(depth\_orig,re\_df,re\_df}\SpecialCharTok{$}\NormalTok{depth\_group)}
\CommentTok{\# rasterize}
\NormalTok{depth\_filled }\OtherTok{\textless{}{-}}\NormalTok{ stars}\SpecialCharTok{::}\FunctionTok{st\_rasterize}\NormalTok{(re\_df) }\SpecialCharTok{\%\textgreater{}\%} \FunctionTok{rast}\NormalTok{()}
\FunctionTok{plot}\NormalTok{(depth\_filled)}
\end{Highlighting}
\end{Shaded}

\pandocbounded{\includegraphics[keepaspectratio]{day4_practical_6_files/figure-pdf/unnamed-chunk-14-1.pdf}}

\end{tcolorbox}

We have now fit the model and want to check the results.

\begin{tcolorbox}[enhanced jigsaw, coltitle=black, rightrule=.15mm, bottomrule=.15mm, left=2mm, arc=.35mm, bottomtitle=1mm, colbacktitle=quarto-callout-warning-color!10!white, leftrule=.75mm, title={Task}, toprule=.15mm, opacitybacktitle=0.6, breakable, colback=white, colframe=quarto-callout-warning-color-frame, toptitle=1mm, opacityback=0, titlerule=0mm]

Is the effect of depth significant? Does it seem important to have a
non-linear model?

Inspect the estimated time effect and

Use the \texttt{predict} function to inspect the estimated spatial
effect.

Use the \texttt{depth\_r} raster to define the points in space where to
predict

\begin{Shaded}
\begin{Highlighting}[]
\NormalTok{pxl }\OtherTok{=} \FunctionTok{st\_as\_sf}\NormalTok{(}\FunctionTok{data.frame}\NormalTok{(}\FunctionTok{crds}\NormalTok{(depth\_r)), }\AttributeTok{coords =} \FunctionTok{c}\NormalTok{(}\StringTok{"x"}\NormalTok{,}\StringTok{"y"}\NormalTok{) ,}
               \AttributeTok{crs  =} \FunctionTok{st\_crs}\NormalTok{(pcod\_sf))}
\end{Highlighting}
\end{Shaded}

Click here to see the solution

\begin{Shaded}
\begin{Highlighting}[]
\CommentTok{\# covariate effect}

\NormalTok{p\_cov }\OtherTok{=}\NormalTok{ fit\_spat}\SpecialCharTok{$}\NormalTok{summary.random}\SpecialCharTok{$}\NormalTok{covariate }\SpecialCharTok{\%\textgreater{}\%}
  \FunctionTok{ggplot}\NormalTok{() }\SpecialCharTok{+} \FunctionTok{geom\_ribbon}\NormalTok{(}\FunctionTok{aes}\NormalTok{(ID, }\AttributeTok{ymin =} \StringTok{\textasciigrave{}}\AttributeTok{0.025quant}\StringTok{\textasciigrave{}}\NormalTok{, }\AttributeTok{ymax =} \StringTok{\textasciigrave{}}\AttributeTok{0.975quant}\StringTok{\textasciigrave{}}\NormalTok{ )) }\SpecialCharTok{+}
  \FunctionTok{geom\_line}\NormalTok{(}\FunctionTok{aes}\NormalTok{(ID,mean)) }\SpecialCharTok{+} \FunctionTok{ggtitle}\NormalTok{(}\StringTok{"Covariate effect"}\NormalTok{)}

\CommentTok{\# time effect}
\NormalTok{p\_time }\OtherTok{=}\NormalTok{ fit\_spat}\SpecialCharTok{$}\NormalTok{summary.random}\SpecialCharTok{$}\NormalTok{trend }\SpecialCharTok{\%\textgreater{}\%}
  \FunctionTok{ggplot}\NormalTok{() }\SpecialCharTok{+} \FunctionTok{geom\_errorbar}\NormalTok{(}\FunctionTok{aes}\NormalTok{(ID, }\AttributeTok{ymin =} \StringTok{\textasciigrave{}}\AttributeTok{0.025quant}\StringTok{\textasciigrave{}}\NormalTok{, }\AttributeTok{ymax =} \StringTok{\textasciigrave{}}\AttributeTok{0.975quant}\StringTok{\textasciigrave{}}\NormalTok{ )) }\SpecialCharTok{+}
  \FunctionTok{geom\_point}\NormalTok{(}\FunctionTok{aes}\NormalTok{(ID,mean)) }\SpecialCharTok{+} \FunctionTok{ggtitle}\NormalTok{(}\StringTok{"Time effect"}\NormalTok{)}

\CommentTok{\#space effect}
\NormalTok{pred\_space }\OtherTok{=} \FunctionTok{predict}\NormalTok{(fit\_spat, pxl, }\SpecialCharTok{\textasciitilde{}}\NormalTok{ space)}

\NormalTok{p\_space\_mean }\OtherTok{=} \FunctionTok{ggplot}\NormalTok{() }\SpecialCharTok{+} \FunctionTok{gg}\NormalTok{(pred\_space, }\FunctionTok{aes}\NormalTok{(}\AttributeTok{color =}\NormalTok{ mean))}
\NormalTok{p\_space\_sd }\OtherTok{=} \FunctionTok{ggplot}\NormalTok{() }\SpecialCharTok{+} \FunctionTok{gg}\NormalTok{(pred\_space, }\FunctionTok{aes}\NormalTok{(}\AttributeTok{color =}\NormalTok{ sd))}
\end{Highlighting}
\end{Shaded}

\end{tcolorbox}

\paragraph{Model 2}\label{model-2}

\begin{enumerate}
\def\labelenumi{\arabic{enumi}.}
\tightlist
\item
  \textbf{Model 2 - spatiotemporal field} We consider a separable space
  time model with a linear predictor given by:
\end{enumerate}

\[
\eta(s,t) = \beta_0 + f_1(\text{depth}(s)) + \omega(s,t)
\]

\begin{itemize}
\tightlist
\item
  \(f_1(\text{depth}(s))\) is a smooth covariate effect of depth (RW2)
\item
  \(\omega(s,t)\) is a space-time Matérn spatial field with AR1 time
  component \[
  \omega(s,t) = \phi\ \omega(s,t-1) + \epsilon(s),\qquad \epsilon(s)\sim\text{GF}(\sigma_{\epsilon},\rho_{\epsilon})
  \]
\end{itemize}

\begin{tcolorbox}[enhanced jigsaw, coltitle=black, rightrule=.15mm, bottomrule=.15mm, left=2mm, arc=.35mm, bottomtitle=1mm, colbacktitle=quarto-callout-warning-color!10!white, leftrule=.75mm, title={Task}, toprule=.15mm, opacitybacktitle=0.6, breakable, colback=white, colframe=quarto-callout-warning-color-frame, toptitle=1mm, opacityback=0, titlerule=0mm]

Implement this second model in \texttt{inlabru}

\begin{enumerate}
\def\labelenumi{\arabic{enumi}.}
\tightlist
\item
  Define the components For the AR1 model use this following PC prior
  for the correlation parameter \(\phi\)
\end{enumerate}

\begin{Shaded}
\begin{Highlighting}[]
\CommentTok{\# PC prior for AR(1) correlation parameter}
\NormalTok{h.spec }\OtherTok{\textless{}{-}} \FunctionTok{list}\NormalTok{(}\AttributeTok{rho =} \FunctionTok{list}\NormalTok{(}\AttributeTok{prior =} \StringTok{\textquotesingle{}pc.cor0\textquotesingle{}}\NormalTok{, }\AttributeTok{param =} \FunctionTok{c}\NormalTok{(}\FloatTok{0.5}\NormalTok{, }\FloatTok{0.1}\NormalTok{)))}
\end{Highlighting}
\end{Shaded}

\begin{enumerate}
\def\labelenumi{\arabic{enumi}.}
\setcounter{enumi}{1}
\tightlist
\item
  Define the formula
\item
  Define the likelihood model using the \texttt{bru\_obs()} function
\item
  Run the model (This model can take a couple of minutes to run)
\end{enumerate}

Click here to see the solution

\begin{Shaded}
\begin{Highlighting}[]
\CommentTok{\# Model components}
\NormalTok{cmp\_spat\_ar1 }\OtherTok{=} \ErrorTok{\textasciitilde{}} \FunctionTok{Intercept}\NormalTok{(}\DecValTok{1}\NormalTok{) }\SpecialCharTok{+} 
  \FunctionTok{covariate}\NormalTok{(depth\_filled}\SpecialCharTok{$}\NormalTok{depth\_group, }\AttributeTok{model =} \StringTok{"rw2"}\NormalTok{, }\AttributeTok{scale.model =} \ConstantTok{TRUE}\NormalTok{)}\SpecialCharTok{+}
  \FunctionTok{space\_time}\NormalTok{(geometry,}
        \AttributeTok{group =}\NormalTok{ time\_idx ,}
        \AttributeTok{model =}\NormalTok{ spde\_model,}
        \AttributeTok{control.group =} \FunctionTok{list}\NormalTok{(}\AttributeTok{model =} \StringTok{\textquotesingle{}ar1\textquotesingle{}}\NormalTok{,}\AttributeTok{hyper =}\NormalTok{ h.spec))}

\CommentTok{\# Linear predictor}
\NormalTok{formula\_spat\_ar1 }\OtherTok{=}\NormalTok{ present }\SpecialCharTok{\textasciitilde{}}\NormalTok{ .}

\CommentTok{\# Observational model}
\NormalTok{lik\_spat\_ar1 }\OtherTok{=} \FunctionTok{bru\_obs}\NormalTok{(}\AttributeTok{formula =}\NormalTok{ formula\_spat\_ar1, }
              \AttributeTok{data =}\NormalTok{ pcod\_sf, }
              \AttributeTok{family =} \StringTok{"binomial"}\NormalTok{)}

\CommentTok{\# Fit Model }
\NormalTok{fit\_spat\_ar1 }\OtherTok{=} \FunctionTok{bru}\NormalTok{(cmp\_spat\_ar1,lik\_spat\_ar1)}
\end{Highlighting}
\end{Shaded}

\end{tcolorbox}

Now we want to check the results

\begin{tcolorbox}[enhanced jigsaw, coltitle=black, rightrule=.15mm, bottomrule=.15mm, left=2mm, arc=.35mm, bottomtitle=1mm, colbacktitle=quarto-callout-warning-color!10!white, leftrule=.75mm, title={Task}, toprule=.15mm, opacitybacktitle=0.6, breakable, colback=white, colframe=quarto-callout-warning-color-frame, toptitle=1mm, opacityback=0, titlerule=0mm]

What is the estimated parameter \(phi\) in the auto-regressive part of
the model?

Check the effect of the covariate.

Use the \texttt{predict} function to inspect the estimated probability
of presence. Use the same prediction points as before, but here you also
need to use the `

Click here to see the solution

\begin{Shaded}
\begin{Highlighting}[]
\CommentTok{\# autoregressive effect}

\CommentTok{\#fit\_spat\_ar1$summary.hyperpar}

\CommentTok{\#covariate}
\NormalTok{p\_cov\_ar1 }\OtherTok{=}\NormalTok{ fit\_spat\_ar1}\SpecialCharTok{$}\NormalTok{summary.random}\SpecialCharTok{$}\NormalTok{covariate }\SpecialCharTok{\%\textgreater{}\%}
  \FunctionTok{ggplot}\NormalTok{() }\SpecialCharTok{+} \FunctionTok{geom\_ribbon}\NormalTok{(}\FunctionTok{aes}\NormalTok{(ID, }\AttributeTok{ymin =} \StringTok{\textasciigrave{}}\AttributeTok{0.025quant}\StringTok{\textasciigrave{}}\NormalTok{, }\AttributeTok{ymax =} \StringTok{\textasciigrave{}}\AttributeTok{0.975quant}\StringTok{\textasciigrave{}}\NormalTok{ )) }\SpecialCharTok{+}
  \FunctionTok{geom\_line}\NormalTok{(}\FunctionTok{aes}\NormalTok{(ID,mean)) }\SpecialCharTok{+} \FunctionTok{ggtitle}\NormalTok{(}\StringTok{"Covariate effect"}\NormalTok{)}


\CommentTok{\#space{-}time effect}

\NormalTok{inv\_logit }\OtherTok{=} \ControlFlowTok{function}\NormalTok{(x)\{ }\FunctionTok{exp}\NormalTok{(x) }\SpecialCharTok{/}\NormalTok{ (}\DecValTok{1} \SpecialCharTok{+} \FunctionTok{exp}\NormalTok{(x))\}}

\NormalTok{pxl\_all }\OtherTok{=} \FunctionTok{fm\_cprod}\NormalTok{(pxl, }\FunctionTok{data.frame}\NormalTok{(}\AttributeTok{time\_idx =} \DecValTok{1}\SpecialCharTok{:}\DecValTok{3}\NormalTok{))}
\NormalTok{pred\_space\_ar1 }\OtherTok{=} \FunctionTok{predict}\NormalTok{(fit\_spat\_ar1, pxl\_all, }\SpecialCharTok{\textasciitilde{}}\FunctionTok{data.frame}\NormalTok{(}\AttributeTok{logit\_prob =}\NormalTok{ Intercept  }\SpecialCharTok{+} 
\NormalTok{                                                               covariate  }\SpecialCharTok{+} 
\NormalTok{                                                               space\_time ,}
                                                            \AttributeTok{prob =} \FunctionTok{inv\_logit}\NormalTok{(Intercept  }\SpecialCharTok{+} 
\NormalTok{                                                               covariate  }\SpecialCharTok{+} 
\NormalTok{                                                               space\_time)))}


\NormalTok{p\_ar1 }\OtherTok{=}\NormalTok{ pred\_space\_ar1}\SpecialCharTok{$}\NormalTok{prob }\SpecialCharTok{\%\textgreater{}\%} \FunctionTok{ggplot}\NormalTok{() }\SpecialCharTok{+} \FunctionTok{geom\_sf}\NormalTok{(}\FunctionTok{aes}\NormalTok{(}\AttributeTok{color =}\NormalTok{ mean)) }\SpecialCharTok{+} 
  \FunctionTok{facet\_wrap}\NormalTok{(.}\SpecialCharTok{\textasciitilde{}}\NormalTok{time\_idx) }\SpecialCharTok{+} \FunctionTok{scale\_color\_scico}\NormalTok{(}\AttributeTok{direction =} \SpecialCharTok{{-}}\DecValTok{1}\NormalTok{) }\SpecialCharTok{+}
\NormalTok{  theme\_map}
\end{Highlighting}
\end{Shaded}

\end{tcolorbox}

\subsubsection{Model Comparison}\label{model-comparison}

Now we want to use the WAIC, DIC and MLIK to compare the models

\begin{tcolorbox}[enhanced jigsaw, coltitle=black, rightrule=.15mm, bottomrule=.15mm, left=2mm, arc=.35mm, bottomtitle=1mm, colbacktitle=quarto-callout-warning-color!10!white, leftrule=.75mm, title={Task}, toprule=.15mm, opacitybacktitle=0.6, breakable, colback=white, colframe=quarto-callout-warning-color-frame, toptitle=1mm, opacityback=0, titlerule=0mm]

Compare the scores, what is your conclusion?

Click here to see the solution

\begin{Shaded}
\begin{Highlighting}[]
\NormalTok{out}\OtherTok{=} \FunctionTok{data.frame}\NormalTok{(}\AttributeTok{Model =} \FunctionTok{c}\NormalTok{(}\StringTok{"Model 1"}\NormalTok{, }\StringTok{"Model 2"}\NormalTok{),}
  \AttributeTok{DIC =} \FunctionTok{c}\NormalTok{(fit\_spat}\SpecialCharTok{$}\NormalTok{dic}\SpecialCharTok{$}\NormalTok{dic, fit\_spat\_ar1}\SpecialCharTok{$}\NormalTok{dic}\SpecialCharTok{$}\NormalTok{dic),}
  \AttributeTok{WAIC =} \FunctionTok{c}\NormalTok{(fit\_spat}\SpecialCharTok{$}\NormalTok{waic}\SpecialCharTok{$}\NormalTok{waic, fit\_spat\_ar1}\SpecialCharTok{$}\NormalTok{waic}\SpecialCharTok{$}\NormalTok{waic),}
  \AttributeTok{MLIK =} \FunctionTok{c}\NormalTok{(fit\_spat}\SpecialCharTok{$}\NormalTok{mlik[}\DecValTok{1}\NormalTok{], fit\_spat\_ar1}\SpecialCharTok{$}\NormalTok{mlik[}\DecValTok{1}\NormalTok{]))}
\end{Highlighting}
\end{Shaded}

\end{tcolorbox}

\subsection{Model check and
comparison}\label{model-check-and-comparison}

\subsection{Model Checking for Linear
Models}\label{model-checking-for-linear-models}

In this exercise we will:

\begin{itemize}
\tightlist
\item
  Learn about some model assessments techniques available in INLA
\item
  Conduct posterior predictive model checking
\end{itemize}

Libraries to load:

\begin{Shaded}
\begin{Highlighting}[]
\FunctionTok{library}\NormalTok{(dplyr)}
\FunctionTok{library}\NormalTok{(tidyr)}
\FunctionTok{library}\NormalTok{(INLA)}
\FunctionTok{library}\NormalTok{(ggplot2)}
\FunctionTok{library}\NormalTok{(patchwork)}
\FunctionTok{library}\NormalTok{(inlabru)     }
\end{Highlighting}
\end{Shaded}

Recall a simple linear regression model with Gaussian observations

\[
y_i\sim\mathcal{N}(\mu_i, \sigma^2), \qquad i = 1,\dots,N
\]

where \(\sigma^2\) is the observation error, and the mean parameter
\(\mu_i\) is linked to the linear predictor through an identity
function:

\[
\eta_i = \mu_i = \beta_0 + \beta_1 x_i
\] where \(x_i\) is a covariate and \(\beta_0, \beta_1\) are parameters
to be estimated.

\subsubsection{Simulate example data}\label{simulate-example-data}

We simulate data from a simple linear regression model

\begin{Shaded}
\begin{Highlighting}[]
\NormalTok{beta }\OtherTok{=} \FunctionTok{c}\NormalTok{(}\DecValTok{2}\NormalTok{,}\FloatTok{0.5}\NormalTok{)}
\NormalTok{sd\_error }\OtherTok{=} \FloatTok{0.1}

\NormalTok{n }\OtherTok{=} \DecValTok{100}
\NormalTok{x }\OtherTok{=} \FunctionTok{rnorm}\NormalTok{(n)}
\NormalTok{y }\OtherTok{=}\NormalTok{ beta[}\DecValTok{1}\NormalTok{] }\SpecialCharTok{+}\NormalTok{ beta[}\DecValTok{2}\NormalTok{] }\SpecialCharTok{*}\NormalTok{ x }\SpecialCharTok{+} \FunctionTok{rnorm}\NormalTok{(n, }\AttributeTok{sd =}\NormalTok{ sd\_error)}

\NormalTok{df }\OtherTok{=} \FunctionTok{data.frame}\NormalTok{(}\AttributeTok{y =}\NormalTok{ y, }\AttributeTok{x =}\NormalTok{ x)  }
\end{Highlighting}
\end{Shaded}

\subsubsection{\texorpdfstring{Fitting the linear regression model with
\texttt{inlabru}}{Fitting the linear regression model with inlabru}}\label{fitting-the-linear-regression-model-with-inlabru}

Now we fit a simple linear regression model in \texttt{inalbru} by
defining (1) the model components, (2) the linear predictor and (3) the
likelihood.

\begin{Shaded}
\begin{Highlighting}[]
\CommentTok{\# Model components}
\NormalTok{cmp }\OtherTok{=}  \ErrorTok{\textasciitilde{}} \SpecialCharTok{{-}}\DecValTok{1} \SpecialCharTok{+} \FunctionTok{beta\_0}\NormalTok{(}\DecValTok{1}\NormalTok{) }\SpecialCharTok{+} \FunctionTok{beta\_1}\NormalTok{(x, }\AttributeTok{model =} \StringTok{"linear"}\NormalTok{)}
\CommentTok{\# Linear predictor}
\NormalTok{formula }\OtherTok{=}\NormalTok{ y }\SpecialCharTok{\textasciitilde{}}\NormalTok{ Intercept }\SpecialCharTok{+}\NormalTok{ beta\_1}
\CommentTok{\# Observational model likelihood}
\NormalTok{lik }\OtherTok{=}  \FunctionTok{bru\_obs}\NormalTok{(}\AttributeTok{formula =}\NormalTok{ y }\SpecialCharTok{\textasciitilde{}}\NormalTok{.,}
            \AttributeTok{family =} \StringTok{"gaussian"}\NormalTok{,}
            \AttributeTok{data =}\NormalTok{ df)}
\CommentTok{\# Fit the Model}
\NormalTok{fit.lm }\OtherTok{=} \FunctionTok{bru}\NormalTok{(cmp, lik)}
\end{Highlighting}
\end{Shaded}

\subsubsection{Residuals analysis}\label{residuals-analysis}

A common way for model diagnostics in regression analysis is by checking
residual plots. In a Bayesian setting residuals can be defined in
multiple ways depending on how you account for posterior uncertainty.
Here, we will adopt a Bayesian approach by generating samples from the
posterior distribution of the model parameters and then draw samples
from the residuals defined as:

\[
r_i = y_i - x_i^T\beta
\]

We can use the \texttt{predict} function to achieve this:

\begin{Shaded}
\begin{Highlighting}[]
\NormalTok{res\_samples }\OtherTok{\textless{}{-}} \FunctionTok{predict}\NormalTok{(}
\NormalTok{  fit.lm,         }\CommentTok{\# the fitted model}
\NormalTok{  df,             }\CommentTok{\# the original data set}
  \SpecialCharTok{\textasciitilde{}} \FunctionTok{data.frame}\NormalTok{(   }
    \AttributeTok{res =}\NormalTok{ y}\SpecialCharTok{{-}}\NormalTok{(beta\_0 }\SpecialCharTok{+}\NormalTok{ beta\_1)  }\CommentTok{\# compute the residuals}
\NormalTok{  ),}
  \AttributeTok{n.samples =} \DecValTok{1000}   \CommentTok{\# draw 1000 samples}
\NormalTok{)}
\end{Highlighting}
\end{Shaded}

The resulting data frame contains the posterior draw of the residuals
mean for which we can produce some diagnostics plots , e.g.

\begin{Shaded}
\begin{Highlighting}[]
\FunctionTok{ggplot}\NormalTok{(res\_samples,}\FunctionTok{aes}\NormalTok{(}\AttributeTok{y=}\NormalTok{mean,}\AttributeTok{x=}\DecValTok{1}\SpecialCharTok{:}\DecValTok{100}\NormalTok{))}\SpecialCharTok{+}\FunctionTok{geom\_point}\NormalTok{() }\SpecialCharTok{+}
\FunctionTok{ggplot}\NormalTok{(res\_samples,}\FunctionTok{aes}\NormalTok{(}\AttributeTok{y=}\NormalTok{mean,}\AttributeTok{x=}\NormalTok{x))}\SpecialCharTok{+}\FunctionTok{geom\_point}\NormalTok{()}
\end{Highlighting}
\end{Shaded}

\begin{figure}[H]

{\centering \pandocbounded{\includegraphics[keepaspectratio]{day4_practical_6_files/figure-pdf/unnamed-chunk-47-1.pdf}}

}

\caption{Bayesian residual plots: the left panel is the residual index
plot; the right panel is the plot of the residual versus the covariate
x}

\end{figure}%

We can also compare these against the theoretical quantiles of the
Normal distribution as follows:

\begin{Shaded}
\begin{Highlighting}[]
\FunctionTok{arrange}\NormalTok{(res\_samples, mean) }\SpecialCharTok{\%\textgreater{}\%}
  \FunctionTok{mutate}\NormalTok{(}\AttributeTok{theortical\_quantiles =} \FunctionTok{qnorm}\NormalTok{(}\DecValTok{1}\SpecialCharTok{:}\DecValTok{100} \SpecialCharTok{/}\NormalTok{ (}\DecValTok{1}\SpecialCharTok{+}\DecValTok{100}\NormalTok{))) }\SpecialCharTok{\%\textgreater{}\%}
  \FunctionTok{ggplot}\NormalTok{(}\FunctionTok{aes}\NormalTok{(}\AttributeTok{x=}\NormalTok{theortical\_quantiles,}\AttributeTok{y=}\NormalTok{ mean)) }\SpecialCharTok{+} 
  \FunctionTok{geom\_ribbon}\NormalTok{(}\FunctionTok{aes}\NormalTok{(}\AttributeTok{ymin =}\NormalTok{ q0}\FloatTok{.025}\NormalTok{, }\AttributeTok{ymax =}\NormalTok{ q0}\FloatTok{.975}\NormalTok{), }\AttributeTok{fill =} \StringTok{"grey70"}\NormalTok{)}\SpecialCharTok{+}
  \FunctionTok{geom\_abline}\NormalTok{(}\AttributeTok{intercept =} \FunctionTok{mean}\NormalTok{(res\_samples}\SpecialCharTok{$}\NormalTok{mean),}
              \AttributeTok{slope =} \FunctionTok{sd}\NormalTok{(res\_samples}\SpecialCharTok{$}\NormalTok{mean)) }\SpecialCharTok{+}
  \FunctionTok{geom\_point}\NormalTok{() }\SpecialCharTok{+}
  \FunctionTok{labs}\NormalTok{(}\AttributeTok{x =} \StringTok{"Theoretical Quantiles (Normal)"}\NormalTok{,}
       \AttributeTok{y=} \StringTok{"Sample Quantiles (Residuals)"}\NormalTok{) }
\end{Highlighting}
\end{Shaded}

\begin{center}
\pandocbounded{\includegraphics[keepaspectratio]{day4_practical_6_files/figure-pdf/unnamed-chunk-48-1.pdf}}
\end{center}

\subsubsection{Posterior Predictive
Checks}\label{posterior-predictive-checks}

Now, instead of generating samples from the mean, we will account for
the observational process uncertainty by:

\begin{enumerate}
\def\labelenumi{\arabic{enumi}.}
\tightlist
\item
  Sampling \(y^{1k}_i\sim\pi(y_i|\mathbf{y})\)
  \(k = 1,\dots,M;~i = 1,\ldots,100\) using \texttt{generate()} (here we
  will draw \(M=500\) samples)
\end{enumerate}

\begin{Shaded}
\begin{Highlighting}[]
\NormalTok{samples }\OtherTok{=}  \FunctionTok{generate}\NormalTok{(fit.lm, df,}
  \AttributeTok{formula =} \SpecialCharTok{\textasciitilde{}}\NormalTok{ \{}
\NormalTok{    mu }\OtherTok{\textless{}{-}}\NormalTok{ (beta\_0 }\SpecialCharTok{+}\NormalTok{ beta\_1)}
\NormalTok{    sd }\OtherTok{\textless{}{-}} \FunctionTok{sqrt}\NormalTok{(}\DecValTok{1} \SpecialCharTok{/}\NormalTok{ Precision\_for\_the\_Gaussian\_observations)}
    \FunctionTok{rnorm}\NormalTok{(}\DecValTok{100}\NormalTok{, }\AttributeTok{mean =}\NormalTok{ mu, }\AttributeTok{sd =}\NormalTok{ sd)}
\NormalTok{  \},}
  \AttributeTok{n.samples =} \DecValTok{500}
\NormalTok{) }
\end{Highlighting}
\end{Shaded}

\begin{enumerate}
\def\labelenumi{\arabic{enumi}.}
\setcounter{enumi}{1}
\tightlist
\item
  Comparing some summaries of the simulated data with the one of the
  observed one
\end{enumerate}

Here we compare (i) the estimated posterior densities
\(\hat{\pi}^k(y|\mathbf{y})\) with the estimated data density and (ii)
the samples means and 95\% credible intervals against the observations.

\begin{Shaded}
\begin{Highlighting}[]
\CommentTok{\# Tidy format for plotting}
\NormalTok{samples\_long }\OtherTok{=} \FunctionTok{data.frame}\NormalTok{(samples) }\SpecialCharTok{\%\textgreater{}\%} 
  \FunctionTok{mutate}\NormalTok{(}\AttributeTok{id =} \DecValTok{1}\SpecialCharTok{:}\DecValTok{100}\NormalTok{) }\SpecialCharTok{\%\textgreater{}\%} \CommentTok{\# i{-}th observation}
  \FunctionTok{pivot\_longer}\NormalTok{(}\SpecialCharTok{{-}}\NormalTok{id)}

\CommentTok{\# compute the mean and quantiles for the samples}
\NormalTok{draws\_summaries }\OtherTok{=} \FunctionTok{data.frame}\NormalTok{(}\AttributeTok{mean\_samples =} \FunctionTok{apply}\NormalTok{(samples,}\DecValTok{1}\NormalTok{,mean),}
\AttributeTok{q25 =} \FunctionTok{apply}\NormalTok{(samples,}\DecValTok{1}\NormalTok{,}\ControlFlowTok{function}\NormalTok{(x)}\FunctionTok{quantile}\NormalTok{(x,}\FloatTok{0.025}\NormalTok{)),  }
\AttributeTok{q975 =} \FunctionTok{apply}\NormalTok{(samples,}\DecValTok{1}\NormalTok{,}\ControlFlowTok{function}\NormalTok{(x)}\FunctionTok{quantile}\NormalTok{(x,}\FloatTok{0.975}\NormalTok{)),}
\AttributeTok{observations =}\NormalTok{ df}\SpecialCharTok{$}\NormalTok{y)  }

\NormalTok{p1 }\OtherTok{=} \FunctionTok{ggplot}\NormalTok{() }\SpecialCharTok{+} \FunctionTok{geom\_density}\NormalTok{(}\AttributeTok{data =}\NormalTok{ samples\_long, }
                        \FunctionTok{aes}\NormalTok{(value, }\AttributeTok{group =}\NormalTok{ name),  }\AttributeTok{color =} \StringTok{"\#E69F00"}\NormalTok{) }\SpecialCharTok{+}
  \FunctionTok{geom\_density}\NormalTok{(}\AttributeTok{data =}\NormalTok{ df, }\FunctionTok{aes}\NormalTok{(y))  }\SpecialCharTok{+}
  \FunctionTok{xlab}\NormalTok{(}\StringTok{""}\NormalTok{) }\SpecialCharTok{+} \FunctionTok{ylab}\NormalTok{(}\StringTok{""}\NormalTok{) }

\NormalTok{p2 }\OtherTok{=} \FunctionTok{ggplot}\NormalTok{(draws\_summaries,}\FunctionTok{aes}\NormalTok{(}\AttributeTok{y=}\NormalTok{mean\_samples,}\AttributeTok{x=}\NormalTok{observations))}\SpecialCharTok{+}
  \FunctionTok{geom\_errorbar}\NormalTok{(}\FunctionTok{aes}\NormalTok{(}\AttributeTok{ymin =}\NormalTok{ q25,}
                   \AttributeTok{ymax =}\NormalTok{ q975), }
               \AttributeTok{alpha =} \FloatTok{0.5}\NormalTok{, }\AttributeTok{color =} \StringTok{"grey50"}\NormalTok{)}\SpecialCharTok{+}
\FunctionTok{geom\_point}\NormalTok{()}\SpecialCharTok{+}\FunctionTok{geom\_abline}\NormalTok{(}\AttributeTok{slope =} \DecValTok{1}\NormalTok{,}\AttributeTok{intercept =} \DecValTok{0}\NormalTok{,}\AttributeTok{lty=}\DecValTok{2}\NormalTok{)}\SpecialCharTok{+}\FunctionTok{labs}\NormalTok{()}

\NormalTok{p1 }\SpecialCharTok{+}\NormalTok{p2}
\end{Highlighting}
\end{Shaded}

\pandocbounded{\includegraphics[keepaspectratio]{day4_practical_6_files/figure-pdf/unnamed-chunk-50-1.pdf}}

\subsection{GLM model checking}\label{sec-linmodel}

In this exercise we will:

\begin{itemize}
\tightlist
\item
  Learn about some model assessments techniques available in INLA
\item
  Conduct posterior predictive model checking using CPO and PIT
\end{itemize}

Libraries to load:

\begin{Shaded}
\begin{Highlighting}[]
\FunctionTok{library}\NormalTok{(dplyr)}
\FunctionTok{library}\NormalTok{(INLA)}
\FunctionTok{library}\NormalTok{(ggplot2)}
\FunctionTok{library}\NormalTok{(patchwork)}
\FunctionTok{library}\NormalTok{(inlabru)     }
\end{Highlighting}
\end{Shaded}

In this exercise, we will use data on horseshoe crabs (\emph{Limulus
polyphemus}) where the number of satellites males surrounding a breeding
female are counted along with the female's color and carapace width.

A possible model to study the factors that affect the number of
satellites for female crabs is

\[
\begin{aligned}
y_i&\sim\mathrm{Poisson}(\mu_i), \qquad i = 1,\dots,N \\
\eta_i &= \mu_i = \beta_0 + \beta_1 x_i + \ldots
\end{aligned}
\]

We can explore the conditional means and variances given the female's
color:

\begin{Shaded}
\begin{Highlighting}[]
\NormalTok{crabs }\OtherTok{\textless{}{-}} \FunctionTok{read.csv}\NormalTok{(}\StringTok{"datasets/crabs.csv"}\NormalTok{)}

\CommentTok{\# conditional means and variances}
\NormalTok{crabs }\SpecialCharTok{\%\textgreater{}\%}
  \FunctionTok{summarise}\NormalTok{( }\AttributeTok{Mean =} \FunctionTok{mean}\NormalTok{(satell ),}
             \AttributeTok{Variance =} \FunctionTok{var}\NormalTok{(satell),}
                     \AttributeTok{.by =}\NormalTok{ color)}
\end{Highlighting}
\end{Shaded}

\begin{verbatim}
   color     Mean  Variance
1 medium 3.294737 10.273908
2   dark 2.227273  6.737844
3  light 4.083333  9.719697
4 darker 2.045455 13.093074
\end{verbatim}

The mean of the number of satellites vary by color which gives a good
indication that color might be useful for predicting satellites numbers.
However, notice that the mean is lower than its variance suggesting that
overdispersion might be present and that a negative binomial model would
be more appropriate for the data (we will cover this later).

\textbf{Fitting the model}

First, lets begin fitting the Poisson model above using the carapace's
color and width as predictors. Since, color is a categorical variable in
our model we need to create a dummy variable for it. We can use the
\texttt{model.matrix} function to help us constructing the design matrix
and then append this to our data:

\begin{Shaded}
\begin{Highlighting}[]
\NormalTok{crabs\_df }\OtherTok{=} \FunctionTok{model.matrix}\NormalTok{( }\SpecialCharTok{\textasciitilde{}}\NormalTok{  color , crabs) }\SpecialCharTok{\%\textgreater{}\%}
  \FunctionTok{as.data.frame}\NormalTok{() }\SpecialCharTok{\%\textgreater{}\%}
  \FunctionTok{select}\NormalTok{(}\SpecialCharTok{{-}}\DecValTok{1}\NormalTok{) }\SpecialCharTok{\%\textgreater{}\%}        \CommentTok{\# drop intercept}
  \FunctionTok{bind\_cols}\NormalTok{(crabs) }\SpecialCharTok{\%\textgreater{}\%}  \CommentTok{\# append to original data}
  \FunctionTok{select}\NormalTok{(}\SpecialCharTok{{-}}\NormalTok{color)        }\CommentTok{\# remove original color categorical variable}
\end{Highlighting}
\end{Shaded}

The new data set \texttt{crabs\_df} contains a dummy variable for the
different color categories (\texttt{dark} being the reference category).
Then we can fit the model in \texttt{inlabru} as follows:

\begin{Shaded}
\begin{Highlighting}[]
\NormalTok{cmp }\OtherTok{=}  \ErrorTok{\textasciitilde{}} \SpecialCharTok{{-}}\DecValTok{1} \SpecialCharTok{+} \FunctionTok{beta0}\NormalTok{(}\DecValTok{1}\NormalTok{) }\SpecialCharTok{+}\NormalTok{  colordarker }\SpecialCharTok{+}
\NormalTok{       colorlight }\SpecialCharTok{+}\NormalTok{ colormedium }\SpecialCharTok{+}
       \FunctionTok{w}\NormalTok{(weight, }\AttributeTok{model =} \StringTok{"linear"}\NormalTok{)}

\NormalTok{lik }\OtherTok{=}  \FunctionTok{bru\_obs}\NormalTok{(}\AttributeTok{formula =}\NormalTok{ satell }\SpecialCharTok{\textasciitilde{}}\NormalTok{.,}
            \AttributeTok{family =} \StringTok{"poisson"}\NormalTok{,}
            \AttributeTok{data =}\NormalTok{ crabs\_df)}

\NormalTok{fit\_pois }\OtherTok{=} \FunctionTok{bru}\NormalTok{(cmp, lik)}

\FunctionTok{summary}\NormalTok{(fit\_pois)}
\end{Highlighting}
\end{Shaded}

\begin{verbatim}
inlabru version: 2.13.0.9011 
INLA version: 25.09.19 
Components: 
Latent components:
beta0: main = linear(1)
colordarker: main = linear(colordarker)
colorlight: main = linear(colorlight)
colormedium: main = linear(colormedium)
w: main = linear(weight)
Observation models: 
  Family: 'poisson'
    Tag: <No tag>
    Data class: 'data.frame'
    Response class: 'integer'
    Predictor: satell ~ .
    Additive/Linear: TRUE/TRUE
    Used components: effects[beta0, colordarker, colorlight, colormedium, w], latent[] 
Time used:
    Pre = 0.886, Running = 0.214, Post = 0.00889, Total = 1.11 
Fixed effects:
              mean    sd 0.025quant 0.5quant 0.975quant   mode kld
beta0       -0.501 0.196     -0.885   -0.501     -0.117 -0.501   0
colordarker -0.008 0.180     -0.362   -0.008      0.345 -0.008   0
colorlight   0.445 0.176      0.101    0.445      0.790  0.445   0
colormedium  0.248 0.118      0.017    0.248      0.479  0.248   0
w            0.001 0.000      0.000    0.001      0.001  0.001   0

Deviance Information Criterion (DIC) ...............: 917.12
Deviance Information Criterion (DIC, saturated) ....: 561.74
Effective number of parameters .....................: 5.01

Watanabe-Akaike information criterion (WAIC) ...: 929.70
Effective number of parameters .................: 16.51

Marginal log-Likelihood:  -489.43 
 is computed 
Posterior summaries for the linear predictor and the fitted values are computed
(Posterior marginals needs also 'control.compute=list(return.marginals.predictor=TRUE)')
\end{verbatim}

\subsubsection{Model assessment and model
choice}\label{model-assessment-and-model-choice}

Now that we have fitted the model we would like to carry some model
assessments. In a Bayesian setting, this is often based on posterior
predictive checks. To do so, we will use the CPO and PIT - two commonly
used Bayesian model assessment criteria based on the \textbf{posterior
predictive distribution}.

\begin{tcolorbox}[enhanced jigsaw, coltitle=black, rightrule=.15mm, bottomrule=.15mm, left=2mm, arc=.35mm, bottomtitle=1mm, colbacktitle=quarto-callout-note-color!10!white, leftrule=.75mm, title=\textcolor{quarto-callout-note-color}{\faInfo}\hspace{0.5em}{Posterior predictive model checking}, toprule=.15mm, opacitybacktitle=0.6, breakable, colback=white, colframe=quarto-callout-note-color-frame, toptitle=1mm, opacityback=0, titlerule=0mm]

The posterior predictive distribution for a predicted value \(\hat{y}\)
is

\[
\pi(\hat{y}|\mathbf{y}) = \int_\theta \pi(\hat{y}|\theta)\pi(\theta|\mathbf{y})d\theta.
\]

The probability integral transform (PIT) introduced by Dawid (1984) is
defined for each observation as:

\[
\mathrm{PIT}_i = \pi(\hat{y}_i \leq y_i |\mathbf{y}{-i})
\]

The PIT evaluates how well a model's predicted values match the observed
data distribution. It is computed as the cumulative distribution
function (CDF) of the observed data evaluated at each predicted value.
If the model is well-calibrated, the PIT values should be
\emph{approximately uniformly distributed}. Deviations from this uniform
distribution may indicate issues with model calibration or overfitting.

Another metric we could used to asses the model fit is the conditional
predictive ordinate (CPO) introduced by Pettit (1990), and defined as:

\[
\text{CPO}_i = \pi(y_i| \mathbf{y}{-i})
\]

The CPO measures the density of the observed value of \(y_i\) when model
is fit using all data but \(y_i\). CPO provides a measure of how well
the model predicts each individual observation while taking into account
the rest of the data and the model. \emph{Large values indicate a better
fit} of the model to the data, while small values indicate a bad fitting
of the model

\end{tcolorbox}

To compute PIT and CPO we can either:

\begin{enumerate}
\def\labelenumi{\arabic{enumi}.}
\item
  ask \texttt{inlabru} to compute them by set
  \texttt{options\ =\ list(control.compute\ =\ list(cpo\ =\ TRUE))} in
  the \texttt{bru()} function arguments.
\item
  set this as default in \texttt{inlabru} global option using the
  \texttt{bru\_options\_set} function.
\end{enumerate}

Here we will do the later and re-run the model

\begin{Shaded}
\begin{Highlighting}[]
\FunctionTok{bru\_options\_set}\NormalTok{(}\AttributeTok{control.compute =} \FunctionTok{list}\NormalTok{(}\AttributeTok{cpo =} \ConstantTok{TRUE}\NormalTok{))}

\NormalTok{fit\_pois }\OtherTok{=} \FunctionTok{bru}\NormalTok{(cmp, lik)}
\end{Highlighting}
\end{Shaded}

Now we can produce histograms and QQ plots to assess for uniformity in
the PIT values which can be accessed through
\texttt{inlabru\_model\$cpo\$pit} :

\section{Plot}

\begin{center}
\pandocbounded{\includegraphics[keepaspectratio]{day4_practical_6_files/figure-pdf/unnamed-chunk-67-1.pdf}}
\end{center}

\section{R Code}

\begin{Shaded}
\begin{Highlighting}[]
\NormalTok{fit\_pois}\SpecialCharTok{$}\NormalTok{cpo}\SpecialCharTok{$}\NormalTok{pit }\SpecialCharTok{\%\textgreater{}\%}
  \FunctionTok{hist}\NormalTok{(}\AttributeTok{main =} \StringTok{"Histogram of PIT values"}\NormalTok{)}

\FunctionTok{qqplot}\NormalTok{(}\FunctionTok{qunif}\NormalTok{(}\FunctionTok{ppoints}\NormalTok{(}\FunctionTok{length}\NormalTok{(fit\_pois}\SpecialCharTok{$}\NormalTok{cpo}\SpecialCharTok{$}\NormalTok{pit))),}
\NormalTok{       fit\_pois}\SpecialCharTok{$}\NormalTok{cpo}\SpecialCharTok{$}\NormalTok{pit,}
       \AttributeTok{main =} \StringTok{"Q{-}Q plot for Unif(0,1)"}\NormalTok{,}
       \AttributeTok{xlab =} \StringTok{"Theoretical Quantiles"}\NormalTok{,}
       \AttributeTok{ylab =} \StringTok{"Sample Quantiles"}\NormalTok{)}

\FunctionTok{qqline}\NormalTok{(fit\_pois}\SpecialCharTok{$}\NormalTok{cpo}\SpecialCharTok{$}\NormalTok{pit,}
       \AttributeTok{distribution =} \ControlFlowTok{function}\NormalTok{(p) }\FunctionTok{qunif}\NormalTok{(p),}
       \AttributeTok{prob =} \FunctionTok{c}\NormalTok{(}\FloatTok{0.1}\NormalTok{, }\FloatTok{0.9}\NormalTok{))}
\end{Highlighting}
\end{Shaded}

Both Q-Q plots and histogram of the PIT values suggest a not so great
model fit. For the CPO values, usually the following summary of the CPO
is often used:

\[
-\sum_{i=1}^n \log (\text{CPO}\_i)
\]

This quantities is useful when comparing different models - a smaller
values indicate a better model fit. CPO values can be accessed by typing
\texttt{inlabru\_model\$cpo\$cpo}.

\begin{tcolorbox}[enhanced jigsaw, coltitle=black, rightrule=.15mm, bottomrule=.15mm, left=2mm, arc=.35mm, bottomtitle=1mm, colbacktitle=quarto-callout-warning-color!10!white, leftrule=.75mm, title={Task}, toprule=.15mm, opacitybacktitle=0.6, breakable, colback=white, colframe=quarto-callout-warning-color-frame, toptitle=1mm, opacityback=0, titlerule=0mm]

The model assessment above suggests that a Poisson model might not be
the most appropriate model, likely due to the overdispersion we detected
previously. Fit a Negative binomial to relax the Poisson model
assumption that the conditional mean and variance are equal. Then,
compute the CPO summary statistic and PIT QQ plot to decide which model
gives the better fit.

Take hint

To specify a negative binomial model you only need to change the family
distribution to \texttt{family\ =\ \ "nbinomial"}.

Click here to see the solution

\begin{Shaded}
\begin{Highlighting}[]
\FunctionTok{par}\NormalTok{(}\AttributeTok{mfrow=}\FunctionTok{c}\NormalTok{(}\DecValTok{1}\NormalTok{,}\DecValTok{2}\NormalTok{))}

\CommentTok{\# Fit the negative binomial model}

\NormalTok{lik\_nbinom }\OtherTok{=}  \FunctionTok{bru\_obs}\NormalTok{(}\AttributeTok{formula =}\NormalTok{ satell }\SpecialCharTok{\textasciitilde{}}\NormalTok{.,}
            \AttributeTok{family =} \StringTok{"nbinomial"}\NormalTok{,}
            \AttributeTok{data =}\NormalTok{ crabs\_df)}

\NormalTok{fit\_nbinom }\OtherTok{=} \FunctionTok{bru}\NormalTok{(cmp, lik\_nbinom)}

\CommentTok{\# PIT checks}

\NormalTok{fit\_nbinom}\SpecialCharTok{$}\NormalTok{cpo}\SpecialCharTok{$}\NormalTok{pit }\SpecialCharTok{\%\textgreater{}\%}
  \FunctionTok{hist}\NormalTok{(}\AttributeTok{main =} \StringTok{"Histogram of PIT values"}\NormalTok{)}

\FunctionTok{qqplot}\NormalTok{(}\FunctionTok{qunif}\NormalTok{(}\FunctionTok{ppoints}\NormalTok{(}\FunctionTok{length}\NormalTok{(fit\_nbinom}\SpecialCharTok{$}\NormalTok{cpo}\SpecialCharTok{$}\NormalTok{pit))),}
\NormalTok{       fit\_nbinom}\SpecialCharTok{$}\NormalTok{cpo}\SpecialCharTok{$}\NormalTok{pit,}
       \AttributeTok{main =} \StringTok{"Q{-}Q plot for Unif(0,1)"}\NormalTok{,}
       \AttributeTok{xlab =} \StringTok{"Theoretical Quantiles"}\NormalTok{,}
       \AttributeTok{ylab =} \StringTok{"Sample Quantiles"}\NormalTok{)}

\FunctionTok{qqline}\NormalTok{(fit\_nbinom}\SpecialCharTok{$}\NormalTok{cpo}\SpecialCharTok{$}\NormalTok{pit,}
       \AttributeTok{distribution =} \ControlFlowTok{function}\NormalTok{(p) }\FunctionTok{qunif}\NormalTok{(p),}
       \AttributeTok{prob =} \FunctionTok{c}\NormalTok{(}\FloatTok{0.1}\NormalTok{, }\FloatTok{0.9}\NormalTok{))}
\end{Highlighting}
\end{Shaded}

\pandocbounded{\includegraphics[keepaspectratio]{day4_practical_6_files/figure-pdf/unnamed-chunk-69-1.pdf}}

\begin{Shaded}
\begin{Highlighting}[]
\CommentTok{\# CPO comparison}

\FunctionTok{data.frame}\NormalTok{( }\AttributeTok{CPO =} \FunctionTok{c}\NormalTok{(}\SpecialCharTok{{-}}\FunctionTok{sum}\NormalTok{(}\FunctionTok{log}\NormalTok{(fit\_pois}\SpecialCharTok{$}\NormalTok{cpo}\SpecialCharTok{$}\NormalTok{cpo)),}
                    \SpecialCharTok{{-}}\FunctionTok{sum}\NormalTok{(}\FunctionTok{log}\NormalTok{(fit\_nbinom}\SpecialCharTok{$}\NormalTok{cpo}\SpecialCharTok{$}\NormalTok{cpo))),}
          \AttributeTok{Model =} \FunctionTok{c}\NormalTok{(}\StringTok{"Poisson"}\NormalTok{,}\StringTok{"Negative Binomial"}\NormalTok{))}
\end{Highlighting}
\end{Shaded}

\begin{verbatim}
       CPO             Model
1 465.4061           Poisson
2 379.3340 Negative Binomial
\end{verbatim}

\begin{Shaded}
\begin{Highlighting}[]
\CommentTok{\# Overall, we can see that the negative binomial model provides a better fit to the data.}
\end{Highlighting}
\end{Shaded}

\end{tcolorbox}




\end{document}
