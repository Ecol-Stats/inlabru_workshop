% Options for packages loaded elsewhere
\PassOptionsToPackage{unicode}{hyperref}
\PassOptionsToPackage{hyphens}{url}
\PassOptionsToPackage{dvipsnames,svgnames,x11names}{xcolor}
%
\documentclass[
  letterpaper,
  DIV=11,
  numbers=noendperiod]{scrartcl}

\usepackage{amsmath,amssymb}
\usepackage{iftex}
\ifPDFTeX
  \usepackage[T1]{fontenc}
  \usepackage[utf8]{inputenc}
  \usepackage{textcomp} % provide euro and other symbols
\else % if luatex or xetex
  \usepackage{unicode-math}
  \defaultfontfeatures{Scale=MatchLowercase}
  \defaultfontfeatures[\rmfamily]{Ligatures=TeX,Scale=1}
\fi
\usepackage{lmodern}
\ifPDFTeX\else  
    % xetex/luatex font selection
\fi
% Use upquote if available, for straight quotes in verbatim environments
\IfFileExists{upquote.sty}{\usepackage{upquote}}{}
\IfFileExists{microtype.sty}{% use microtype if available
  \usepackage[]{microtype}
  \UseMicrotypeSet[protrusion]{basicmath} % disable protrusion for tt fonts
}{}
\makeatletter
\@ifundefined{KOMAClassName}{% if non-KOMA class
  \IfFileExists{parskip.sty}{%
    \usepackage{parskip}
  }{% else
    \setlength{\parindent}{0pt}
    \setlength{\parskip}{6pt plus 2pt minus 1pt}}
}{% if KOMA class
  \KOMAoptions{parskip=half}}
\makeatother
\usepackage{xcolor}
\setlength{\emergencystretch}{3em} % prevent overfull lines
\setcounter{secnumdepth}{5}
% Make \paragraph and \subparagraph free-standing
\makeatletter
\ifx\paragraph\undefined\else
  \let\oldparagraph\paragraph
  \renewcommand{\paragraph}{
    \@ifstar
      \xxxParagraphStar
      \xxxParagraphNoStar
  }
  \newcommand{\xxxParagraphStar}[1]{\oldparagraph*{#1}\mbox{}}
  \newcommand{\xxxParagraphNoStar}[1]{\oldparagraph{#1}\mbox{}}
\fi
\ifx\subparagraph\undefined\else
  \let\oldsubparagraph\subparagraph
  \renewcommand{\subparagraph}{
    \@ifstar
      \xxxSubParagraphStar
      \xxxSubParagraphNoStar
  }
  \newcommand{\xxxSubParagraphStar}[1]{\oldsubparagraph*{#1}\mbox{}}
  \newcommand{\xxxSubParagraphNoStar}[1]{\oldsubparagraph{#1}\mbox{}}
\fi
\makeatother

\usepackage{color}
\usepackage{fancyvrb}
\newcommand{\VerbBar}{|}
\newcommand{\VERB}{\Verb[commandchars=\\\{\}]}
\DefineVerbatimEnvironment{Highlighting}{Verbatim}{commandchars=\\\{\}}
% Add ',fontsize=\small' for more characters per line
\usepackage{framed}
\definecolor{shadecolor}{RGB}{241,243,245}
\newenvironment{Shaded}{\begin{snugshade}}{\end{snugshade}}
\newcommand{\AlertTok}[1]{\textcolor[rgb]{0.68,0.00,0.00}{#1}}
\newcommand{\AnnotationTok}[1]{\textcolor[rgb]{0.37,0.37,0.37}{#1}}
\newcommand{\AttributeTok}[1]{\textcolor[rgb]{0.40,0.45,0.13}{#1}}
\newcommand{\BaseNTok}[1]{\textcolor[rgb]{0.68,0.00,0.00}{#1}}
\newcommand{\BuiltInTok}[1]{\textcolor[rgb]{0.00,0.23,0.31}{#1}}
\newcommand{\CharTok}[1]{\textcolor[rgb]{0.13,0.47,0.30}{#1}}
\newcommand{\CommentTok}[1]{\textcolor[rgb]{0.37,0.37,0.37}{#1}}
\newcommand{\CommentVarTok}[1]{\textcolor[rgb]{0.37,0.37,0.37}{\textit{#1}}}
\newcommand{\ConstantTok}[1]{\textcolor[rgb]{0.56,0.35,0.01}{#1}}
\newcommand{\ControlFlowTok}[1]{\textcolor[rgb]{0.00,0.23,0.31}{\textbf{#1}}}
\newcommand{\DataTypeTok}[1]{\textcolor[rgb]{0.68,0.00,0.00}{#1}}
\newcommand{\DecValTok}[1]{\textcolor[rgb]{0.68,0.00,0.00}{#1}}
\newcommand{\DocumentationTok}[1]{\textcolor[rgb]{0.37,0.37,0.37}{\textit{#1}}}
\newcommand{\ErrorTok}[1]{\textcolor[rgb]{0.68,0.00,0.00}{#1}}
\newcommand{\ExtensionTok}[1]{\textcolor[rgb]{0.00,0.23,0.31}{#1}}
\newcommand{\FloatTok}[1]{\textcolor[rgb]{0.68,0.00,0.00}{#1}}
\newcommand{\FunctionTok}[1]{\textcolor[rgb]{0.28,0.35,0.67}{#1}}
\newcommand{\ImportTok}[1]{\textcolor[rgb]{0.00,0.46,0.62}{#1}}
\newcommand{\InformationTok}[1]{\textcolor[rgb]{0.37,0.37,0.37}{#1}}
\newcommand{\KeywordTok}[1]{\textcolor[rgb]{0.00,0.23,0.31}{\textbf{#1}}}
\newcommand{\NormalTok}[1]{\textcolor[rgb]{0.00,0.23,0.31}{#1}}
\newcommand{\OperatorTok}[1]{\textcolor[rgb]{0.37,0.37,0.37}{#1}}
\newcommand{\OtherTok}[1]{\textcolor[rgb]{0.00,0.23,0.31}{#1}}
\newcommand{\PreprocessorTok}[1]{\textcolor[rgb]{0.68,0.00,0.00}{#1}}
\newcommand{\RegionMarkerTok}[1]{\textcolor[rgb]{0.00,0.23,0.31}{#1}}
\newcommand{\SpecialCharTok}[1]{\textcolor[rgb]{0.37,0.37,0.37}{#1}}
\newcommand{\SpecialStringTok}[1]{\textcolor[rgb]{0.13,0.47,0.30}{#1}}
\newcommand{\StringTok}[1]{\textcolor[rgb]{0.13,0.47,0.30}{#1}}
\newcommand{\VariableTok}[1]{\textcolor[rgb]{0.07,0.07,0.07}{#1}}
\newcommand{\VerbatimStringTok}[1]{\textcolor[rgb]{0.13,0.47,0.30}{#1}}
\newcommand{\WarningTok}[1]{\textcolor[rgb]{0.37,0.37,0.37}{\textit{#1}}}

\providecommand{\tightlist}{%
  \setlength{\itemsep}{0pt}\setlength{\parskip}{0pt}}\usepackage{longtable,booktabs,array}
\usepackage{calc} % for calculating minipage widths
% Correct order of tables after \paragraph or \subparagraph
\usepackage{etoolbox}
\makeatletter
\patchcmd\longtable{\par}{\if@noskipsec\mbox{}\fi\par}{}{}
\makeatother
% Allow footnotes in longtable head/foot
\IfFileExists{footnotehyper.sty}{\usepackage{footnotehyper}}{\usepackage{footnote}}
\makesavenoteenv{longtable}
\usepackage{graphicx}
\makeatletter
\newsavebox\pandoc@box
\newcommand*\pandocbounded[1]{% scales image to fit in text height/width
  \sbox\pandoc@box{#1}%
  \Gscale@div\@tempa{\textheight}{\dimexpr\ht\pandoc@box+\dp\pandoc@box\relax}%
  \Gscale@div\@tempb{\linewidth}{\wd\pandoc@box}%
  \ifdim\@tempb\p@<\@tempa\p@\let\@tempa\@tempb\fi% select the smaller of both
  \ifdim\@tempa\p@<\p@\scalebox{\@tempa}{\usebox\pandoc@box}%
  \else\usebox{\pandoc@box}%
  \fi%
}
% Set default figure placement to htbp
\def\fps@figure{htbp}
\makeatother

% load packages
\usepackage{geometry}
\usepackage{xcolor}
\usepackage{eso-pic}
\usepackage{fancyhdr}
\usepackage{sectsty}
\usepackage{fontspec}
\usepackage{titlesec}

%% Set page size with a wider right margin
\geometry{a4paper, total={170mm,257mm}, left=20mm, top=20mm, bottom=20mm, right=50mm}

%% Let's define some colours
\definecolor{light}{HTML}{E6E6FA}
\definecolor{highlight}{HTML}{800080}
\definecolor{dark}{HTML}{330033}

%% Let's add the border on the right hand side 
\AddToShipoutPicture{% 
    \AtPageLowerLeft{% 
        \put(\LenToUnit{\dimexpr\paperwidth-3cm},0){% 
            \color{light}\rule{3cm}{\LenToUnit\paperheight}%
          }%
     }%
     % logo
    \AtPageLowerLeft{% start the bar at the bottom right of the page
        \put(\LenToUnit{\dimexpr\paperwidth-2.25cm},27.2cm){% move it to the top right
            \color{light}\includegraphics[width=1.5cm]{_extensions/nrennie/PrettyPDF/logo.png}
          }%
     }%
}

%% Style the page number
\fancypagestyle{mystyle}{
  \fancyhf{}
  \renewcommand\headrulewidth{0pt}
  \fancyfoot[R]{\thepage}
  \fancyfootoffset{3.5cm}
}
\setlength{\footskip}{20pt}

%% style the chapter/section fonts
\chapterfont{\color{dark}\fontsize{20}{16.8}\selectfont}
\sectionfont{\color{dark}\fontsize{20}{16.8}\selectfont}
\subsectionfont{\color{dark}\fontsize{14}{16.8}\selectfont}
\titleformat{\subsection}
  {\sffamily\Large\bfseries}{\thesection}{1em}{}[{\titlerule[0.8pt]}]
  
% left align title
\makeatletter
\renewcommand{\maketitle}{\bgroup\setlength{\parindent}{0pt}
\begin{flushleft}
  {\sffamily\huge\textbf{\MakeUppercase{\@title}}} \vspace{0.3cm} \newline
  {\Large {\@subtitle}} \newline
  \@author
\end{flushleft}\egroup
}
\makeatother

%% Use some custom fonts
\setsansfont{Ubuntu}[
    Path=_extensions/nrennie/PrettyPDF/Ubuntu/,
    Scale=0.9,
    Extension = .ttf,
    UprightFont=*-Regular,
    BoldFont=*-Bold,
    ItalicFont=*-Italic,
    ]

\setmainfont{Ubuntu}[
    Path=_extensions/nrennie/PrettyPDF/Ubuntu/,
    Scale=0.9,
    Extension = .ttf,
    UprightFont=*-Regular,
    BoldFont=*-Bold,
    ItalicFont=*-Italic,
    ]
\KOMAoption{captions}{tableheading}
\makeatletter
\@ifpackageloaded{tcolorbox}{}{\usepackage[skins,breakable]{tcolorbox}}
\@ifpackageloaded{fontawesome5}{}{\usepackage{fontawesome5}}
\definecolor{quarto-callout-color}{HTML}{909090}
\definecolor{quarto-callout-note-color}{HTML}{0758E5}
\definecolor{quarto-callout-important-color}{HTML}{CC1914}
\definecolor{quarto-callout-warning-color}{HTML}{EB9113}
\definecolor{quarto-callout-tip-color}{HTML}{00A047}
\definecolor{quarto-callout-caution-color}{HTML}{FC5300}
\definecolor{quarto-callout-color-frame}{HTML}{acacac}
\definecolor{quarto-callout-note-color-frame}{HTML}{4582ec}
\definecolor{quarto-callout-important-color-frame}{HTML}{d9534f}
\definecolor{quarto-callout-warning-color-frame}{HTML}{f0ad4e}
\definecolor{quarto-callout-tip-color-frame}{HTML}{02b875}
\definecolor{quarto-callout-caution-color-frame}{HTML}{fd7e14}
\makeatother
\makeatletter
\@ifpackageloaded{caption}{}{\usepackage{caption}}
\AtBeginDocument{%
\ifdefined\contentsname
  \renewcommand*\contentsname{Table of contents}
\else
  \newcommand\contentsname{Table of contents}
\fi
\ifdefined\listfigurename
  \renewcommand*\listfigurename{List of Figures}
\else
  \newcommand\listfigurename{List of Figures}
\fi
\ifdefined\listtablename
  \renewcommand*\listtablename{List of Tables}
\else
  \newcommand\listtablename{List of Tables}
\fi
\ifdefined\figurename
  \renewcommand*\figurename{Figure}
\else
  \newcommand\figurename{Figure}
\fi
\ifdefined\tablename
  \renewcommand*\tablename{Table}
\else
  \newcommand\tablename{Table}
\fi
}
\@ifpackageloaded{float}{}{\usepackage{float}}
\floatstyle{ruled}
\@ifundefined{c@chapter}{\newfloat{codelisting}{h}{lop}}{\newfloat{codelisting}{h}{lop}[chapter]}
\floatname{codelisting}{Listing}
\newcommand*\listoflistings{\listof{codelisting}{List of Listings}}
\makeatother
\makeatletter
\makeatother
\makeatletter
\@ifpackageloaded{caption}{}{\usepackage{caption}}
\@ifpackageloaded{subcaption}{}{\usepackage{subcaption}}
\makeatother
\makeatletter
\@ifpackageloaded{tcolorbox}{}{\usepackage[skins,breakable]{tcolorbox}}
\makeatother
\makeatletter
\@ifundefined{shadecolor}{\definecolor{shadecolor}{rgb}{.97, .97, .97}}{}
\makeatother
\makeatletter
\@ifundefined{codebgcolor}{\definecolor{codebgcolor}{named}{light}}{}
\makeatother
\makeatletter
\ifdefined\Shaded\renewenvironment{Shaded}{\begin{tcolorbox}[sharp corners, breakable, frame hidden, colback={codebgcolor}, enhanced, boxrule=0pt]}{\end{tcolorbox}}\fi
\makeatother

\usepackage{bookmark}

\IfFileExists{xurl.sty}{\usepackage{xurl}}{} % add URL line breaks if available
\urlstyle{same} % disable monospaced font for URLs
\hypersetup{
  pdftitle={Practical 8: Zero-inflation},
  colorlinks=true,
  linkcolor={highlight},
  filecolor={Maroon},
  citecolor={Blue},
  urlcolor={highlight},
  pdfcreator={LaTeX via pandoc}}


\title{Practical 8: Zero-inflation}
\author{}
\date{}

\begin{document}
\maketitle

\pagestyle{mystyle}


\textbf{Aim of this practical:}

In this practical we are going to look at some model comparison and
validation techniques.

\section{ZIP/ZAP and Hurdle Models}\label{zipzap-and-hurdle-models}

\begin{Shaded}
\begin{Highlighting}[]
\FunctionTok{library}\NormalTok{(dplyr)}
\FunctionTok{library}\NormalTok{(ggplot2)}
\FunctionTok{library}\NormalTok{(inlabru)}
\FunctionTok{library}\NormalTok{(INLA)}
\FunctionTok{library}\NormalTok{(terra)}
\FunctionTok{library}\NormalTok{(sf)}
\FunctionTok{library}\NormalTok{(scico)}
\FunctionTok{library}\NormalTok{(magrittr)}
\FunctionTok{library}\NormalTok{(patchwork)}
\FunctionTok{library}\NormalTok{(tidyterra)}


\CommentTok{\# We want to obtain CPO data from the estimations}
\FunctionTok{bru\_options\_set}\NormalTok{(}\AttributeTok{control.compute =} \FunctionTok{list}\NormalTok{(}\AttributeTok{dic =} \ConstantTok{TRUE}\NormalTok{,}
                                       \AttributeTok{waic =} \ConstantTok{TRUE}\NormalTok{,}
                                       \AttributeTok{mlik =} \ConstantTok{TRUE}\NormalTok{,}
                                       \AttributeTok{cpo =} \ConstantTok{TRUE}\NormalTok{))}
\end{Highlighting}
\end{Shaded}

In this practical we are going to work with data with excess zeros. We
will

\begin{itemize}
\item
  \hyperref[sec-prep]{Create count data from a \texttt{gorillas}
  dataset}
\item
  Fit a \hyperref[sec-zip]{zero inflated model}
\item
  Fit a \hyperref[sec-zap]{hurdle model}
\item
  Fit a \hyperref[sec-two-lik]{hurdle model using two likelihoods}
\item
  Fit a \hyperref[sec-two-lik-share]{hurdle model using two likelihoods
  and a shared component}
\end{itemize}

\subsection{Data Preparation}\label{sec-prep}

The following example use the \texttt{gorillas} dataset available in the
\texttt{inlabru} library.

The data give the locations of Gorilla's nests in an area: ::: \{.cell
layout-align=``center''\}

\begin{Shaded}
\begin{Highlighting}[]
\NormalTok{gorillas\_sf }\OtherTok{\textless{}{-}}\NormalTok{ inlabru}\SpecialCharTok{::}\NormalTok{gorillas\_sf}
\NormalTok{nests }\OtherTok{\textless{}{-}}\NormalTok{ gorillas\_sf}\SpecialCharTok{$}\NormalTok{nests}
\NormalTok{boundary }\OtherTok{\textless{}{-}}\NormalTok{ gorillas\_sf}\SpecialCharTok{$}\NormalTok{boundary}

\FunctionTok{ggplot}\NormalTok{() }\SpecialCharTok{+} \FunctionTok{geom\_sf}\NormalTok{(}\AttributeTok{data =}\NormalTok{ nests) }\SpecialCharTok{+}
  \FunctionTok{geom\_sf}\NormalTok{(}\AttributeTok{data =}\NormalTok{ boundary, }\AttributeTok{alpha =} \DecValTok{0}\NormalTok{)}
\end{Highlighting}
\end{Shaded}

\begin{figure}[H]

{\centering \includegraphics[width=0.8\linewidth,height=\textheight,keepaspectratio]{day5_practical_8_files/figure-pdf/nests_loc-1.png}

}

\caption{Location of gorilla nests}

\end{figure}%

:::

The dataset also contains covariates in the form or raster data. We
consider two of them here: ::: \{.cell\}

\begin{Shaded}
\begin{Highlighting}[]
\NormalTok{gcov }\OtherTok{=} \FunctionTok{gorillas\_sf\_gcov}\NormalTok{()}
\NormalTok{elev\_cov }\OtherTok{\textless{}{-}}\NormalTok{ gcov}\SpecialCharTok{$}\NormalTok{elevation}
\NormalTok{dist\_cov }\OtherTok{\textless{}{-}}\NormalTok{  gcov}\SpecialCharTok{$}\NormalTok{waterdist}
\end{Highlighting}
\end{Shaded}

:::

\begin{figure}[H]

{\centering \includegraphics[width=0.8\linewidth,height=\textheight,keepaspectratio]{day5_practical_8_files/figure-pdf/unnamed-chunk-4-1.png}

}

\caption{Covariates}

\end{figure}%

\textbf{Note:} the covariates have been expanded to cover all the nodes
in the mesh.

\begin{center}\rule{0.5\linewidth}{0.5pt}\end{center}

To obtain the count data, we rasterize the species counts to match the
spatial resolution of the covariates available. Then we aggregate the
pixels to a rougher resolution (5x5 pixels in the original covariate
raster dimensions). Finally, we mask regions outside the study area.

In addition we compute the area of each grid cell.

\begin{Shaded}
\begin{Highlighting}[]
\CommentTok{\# Rasterize data}
\NormalTok{counts\_rstr }\OtherTok{\textless{}{-}}
\NormalTok{  terra}\SpecialCharTok{::}\FunctionTok{rasterize}\NormalTok{(}\FunctionTok{vect}\NormalTok{(nests), gcov, }\AttributeTok{fun =}\NormalTok{ sum, }\AttributeTok{background =} \DecValTok{0}\NormalTok{) }\SpecialCharTok{\%\textgreater{}\%}
\NormalTok{  terra}\SpecialCharTok{::}\FunctionTok{aggregate}\NormalTok{(}\AttributeTok{fact =} \DecValTok{5}\NormalTok{, }\AttributeTok{fun =}\NormalTok{ sum) }\SpecialCharTok{\%\textgreater{}\%}
  \FunctionTok{mask}\NormalTok{(}\FunctionTok{vect}\NormalTok{(sf}\SpecialCharTok{::}\FunctionTok{st\_geometry}\NormalTok{(boundary)))}
\FunctionTok{plot}\NormalTok{(counts\_rstr)}
\end{Highlighting}
\end{Shaded}

\begin{figure}[H]

{\centering \includegraphics[width=0.8\linewidth,height=\textheight,keepaspectratio]{day5_practical_8_files/figure-pdf/unnamed-chunk-5-1.png}

}

\caption{Counts of gorilla nests}

\end{figure}%

\begin{Shaded}
\begin{Highlighting}[]
\CommentTok{\# compute cell area}
\NormalTok{counts\_rstr }\OtherTok{\textless{}{-}}\NormalTok{ counts\_rstr }\SpecialCharTok{\%\textgreater{}\%}
  \FunctionTok{cellSize}\NormalTok{(}\AttributeTok{unit =} \StringTok{"km"}\NormalTok{) }\SpecialCharTok{\%\textgreater{}\%}
  \FunctionTok{c}\NormalTok{(counts\_rstr)}
\end{Highlighting}
\end{Shaded}

To create our dataset of counts, we extract also the coordinate of
center point of each raster pixel. In addition we create a column with
presences and one with the pixel area

\begin{Shaded}
\begin{Highlighting}[]
\NormalTok{counts\_df }\OtherTok{\textless{}{-}} \FunctionTok{crds}\NormalTok{(counts\_rstr, }\AttributeTok{df =} \ConstantTok{TRUE}\NormalTok{, }\AttributeTok{na.rm =} \ConstantTok{TRUE}\NormalTok{) }\SpecialCharTok{\%\textgreater{}\%}
  \FunctionTok{bind\_cols}\NormalTok{(}\FunctionTok{values}\NormalTok{(counts\_rstr, }\AttributeTok{mat =} \ConstantTok{TRUE}\NormalTok{, }\AttributeTok{na.rm =} \ConstantTok{TRUE}\NormalTok{)) }\SpecialCharTok{\%\textgreater{}\%}
  \FunctionTok{rename}\NormalTok{(}\AttributeTok{count =}\NormalTok{ sum) }\SpecialCharTok{\%\textgreater{}\%}
  \FunctionTok{mutate}\NormalTok{(}\AttributeTok{present =}\NormalTok{ (count }\SpecialCharTok{\textgreater{}} \DecValTok{0}\NormalTok{) }\SpecialCharTok{*} \DecValTok{1}\NormalTok{L) }\SpecialCharTok{\%\textgreater{}\%}
  \FunctionTok{st\_as\_sf}\NormalTok{(}\AttributeTok{coords =} \FunctionTok{c}\NormalTok{(}\StringTok{"x"}\NormalTok{, }\StringTok{"y"}\NormalTok{), }\AttributeTok{crs =} \FunctionTok{st\_crs}\NormalTok{(nests))}
\end{Highlighting}
\end{Shaded}

We then aggregate the covariates to the same resolution as the nest
counts and scale them.

\begin{Shaded}
\begin{Highlighting}[]
\NormalTok{elev\_cov1 }\OtherTok{\textless{}{-}}\NormalTok{ elev\_cov }\SpecialCharTok{\%\textgreater{}\%} 
\NormalTok{  terra}\SpecialCharTok{::}\FunctionTok{aggregate}\NormalTok{(}\AttributeTok{fact =} \DecValTok{5}\NormalTok{, }\AttributeTok{fun =}\NormalTok{ mean) }\SpecialCharTok{\%\textgreater{}\%} \FunctionTok{scale}\NormalTok{()}
\NormalTok{dist\_cov1 }\OtherTok{\textless{}{-}}\NormalTok{ dist\_cov }\SpecialCharTok{\%\textgreater{}\%} 
\NormalTok{  terra}\SpecialCharTok{::}\FunctionTok{aggregate}\NormalTok{(}\AttributeTok{fact =} \DecValTok{5}\NormalTok{, }\AttributeTok{fun =}\NormalTok{ mean) }\SpecialCharTok{\%\textgreater{}\%} \FunctionTok{scale}\NormalTok{()}
\end{Highlighting}
\end{Shaded}

\begin{figure}

\centering{

\includegraphics[width=0.8\linewidth,height=\textheight,keepaspectratio]{day5_practical_8_files/figure-pdf/fig-covariate-raster-1.png}

}

\caption{\label{fig-covariate-raster}Covariates}

\end{figure}%

\subsubsection{Mesh building}\label{mesh-building}

We now define the mesh and the spde object.

\begin{Shaded}
\begin{Highlighting}[]

\NormalTok{mesh }\OtherTok{\textless{}{-}} \FunctionTok{fm\_mesh\_2d}\NormalTok{(}
  \AttributeTok{loc =} \FunctionTok{st\_as\_sfc}\NormalTok{(counts\_df),}
  \AttributeTok{max.edge =} \FunctionTok{c}\NormalTok{(}\FloatTok{0.5}\NormalTok{, }\DecValTok{1}\NormalTok{),}
  \AttributeTok{crs =} \FunctionTok{st\_crs}\NormalTok{(counts\_df)}
\NormalTok{)}

\NormalTok{matern }\OtherTok{\textless{}{-}} \FunctionTok{inla.spde2.pcmatern}\NormalTok{(mesh,}
  \AttributeTok{prior.sigma =} \FunctionTok{c}\NormalTok{(}\DecValTok{1}\NormalTok{, }\FloatTok{0.01}\NormalTok{),}
  \AttributeTok{prior.range =} \FunctionTok{c}\NormalTok{(}\DecValTok{5}\NormalTok{, }\FloatTok{0.01}\NormalTok{)}
\NormalTok{)}
\end{Highlighting}
\end{Shaded}

\begin{figure}[H]

{\centering \includegraphics[width=0.8\linewidth,height=\textheight,keepaspectratio]{day5_practical_8_files/figure-pdf/unnamed-chunk-9-1.png}

}

\caption{Mesh over the count locations}

\end{figure}%

In our dataset, the number of zeros is quite substantial, and our model
may struggle to account for them adequately. To address this, we should
select a model capable of handling an ``inflated'' number of zeros,
exceeding what a standard Poisson model would imply. For this purpose,
we opt for a ``zero-inflated Poisson model,'' commonly abbreviated as
ZIP.

\subsection{Zero-Inflated model (Type1)}\label{sec-zip}

We fit now a Zero-Inflated model to our data.

The
\href{https://inla.r-inla-download.org/r-inla.org/doc/likelihood/zeroinflated.pdf}{Type
1 Zero-inflated Poisson model} is defined as follows:

\[
\text{Prob}(y\vert\dots)=\pi\times 1_{y=0}+(1-\pi)\times \text{Poisson}(y)
\]

Here, \(\pi=\text{logit}^{-1}(\theta)\)

The expected value and variance for the counts are calculated as:

\begin{equation}\phantomsection\label{eq-mean-zip}{
\begin{gathered}
E(count)=(1-\pi)\lambda \\
Var(count)= (1-\pi)(\lambda+\pi \lambda^2)
\end{gathered}
}\end{equation}

This model has two parameters:

\begin{itemize}
\tightlist
\item
  The probability of excess zero \(\pi\) - This is a
  \emph{hyperparameter} and therefore it is constant
\item
  The mean of the Poisson distribution \(\lambda\). This is linked to
  the linear predictor as: \[
  \eta = E\log(\lambda) = \log(E) + \beta_0 + \beta_1\text{Elevation} + \beta_2\text{Distance } + u
  \] where \(\log(E)\) is an offset (the area of the pixel) that
  accounts for the size of the cell.
\end{itemize}

\begin{tcolorbox}[enhanced jigsaw, colframe=quarto-callout-warning-color-frame, colback=white, colbacktitle=quarto-callout-warning-color!10!white, left=2mm, breakable, toptitle=1mm, toprule=.15mm, coltitle=black, bottomrule=.15mm, titlerule=0mm, leftrule=.75mm, rightrule=.15mm, title={Task}, bottomtitle=1mm, arc=.35mm, opacityback=0, opacitybacktitle=0.6]

Fit a zero-inflated model to the data (\texttt{zeroinflatedpoisson1}) by
completing the following code: ::: \{.cell\}

\begin{Shaded}
\begin{Highlighting}[]
\NormalTok{cmp }\OtherTok{=} \ErrorTok{\textasciitilde{}} \FunctionTok{Intercept}\NormalTok{(}\DecValTok{1}\NormalTok{) }\SpecialCharTok{+} \FunctionTok{elevation}\NormalTok{(...) }\SpecialCharTok{+} \FunctionTok{distance}\NormalTok{(...) }\SpecialCharTok{+} \FunctionTok{space}\NormalTok{(...)}

\NormalTok{lik }\OtherTok{=} \FunctionTok{bru\_obs}\NormalTok{(...,}
    \AttributeTok{E =}\NormalTok{ area)}

\NormalTok{fit\_zip }\OtherTok{\textless{}{-}} \FunctionTok{bru}\NormalTok{(cmp, lik)}
\end{Highlighting}
\end{Shaded}

\end{tcolorbox}

Take hint

The \texttt{E\ =\ area} is an offset that adjusts for the size of each
cell.

Click here to see the solution

\begin{Shaded}
\begin{Highlighting}[]

\NormalTok{cmp }\OtherTok{=} \ErrorTok{\textasciitilde{}} \FunctionTok{Intercept}\NormalTok{(}\DecValTok{1}\NormalTok{) }\SpecialCharTok{+} \FunctionTok{elevation}\NormalTok{(elev\_cov1, }\AttributeTok{model =} \StringTok{"linear"}\NormalTok{) }\SpecialCharTok{+} \FunctionTok{distance}\NormalTok{(dist\_cov1, }\AttributeTok{model =} \StringTok{"linear"}\NormalTok{) }\SpecialCharTok{+} \FunctionTok{space}\NormalTok{(geometry, }\AttributeTok{model =}\NormalTok{ matern)}



\NormalTok{lik }\OtherTok{=} \FunctionTok{bru\_obs}\NormalTok{(}\AttributeTok{formula =}\NormalTok{ count }\SpecialCharTok{\textasciitilde{}}\NormalTok{ .,}
    \AttributeTok{family =} \StringTok{"zeroinflatedpoisson1"}\NormalTok{, }
    \AttributeTok{data =}\NormalTok{ counts\_df,}
    \AttributeTok{E =}\NormalTok{ area)}

\NormalTok{fit\_zip }\OtherTok{\textless{}{-}} \FunctionTok{bru}\NormalTok{(cmp, lik)}
\end{Highlighting}
\end{Shaded}

:::

Once the model is fitted we can look at the results

\begin{tcolorbox}[enhanced jigsaw, colframe=quarto-callout-warning-color-frame, colback=white, colbacktitle=quarto-callout-warning-color!10!white, left=2mm, breakable, toptitle=1mm, toprule=.15mm, coltitle=black, bottomrule=.15mm, titlerule=0mm, leftrule=.75mm, rightrule=.15mm, title={Task}, bottomtitle=1mm, arc=.35mm, opacityback=0, opacitybacktitle=0.6]

Check what the estimated excess zero probaility is.

Use the \texttt{predict()} function to look at the estimated
\(\lambda(s)\) and mean count in Equation~\ref{eq-mean-zip}

Take hint

To get the right name for the hyperparameters to use in the
\texttt{predict()} function, you can use the function
\texttt{bru\_names()}.

Click here to see the solution

\begin{Shaded}
\begin{Highlighting}[]
\CommentTok{\# to check the estimated excess zero probability:}
\CommentTok{\# fit\_zip$summary.hyperpar}

\NormalTok{pred\_zip }\OtherTok{\textless{}{-}} \FunctionTok{predict}\NormalTok{(}
\NormalTok{  fit\_zip, }
\NormalTok{  counts\_df,}
  \SpecialCharTok{\textasciitilde{}}\NormalTok{ \{}
\NormalTok{    pi }\OtherTok{\textless{}{-}}\NormalTok{ zero\_probability\_parameter\_for\_zero\_inflated\_poisson\_1}
\NormalTok{    lambda }\OtherTok{\textless{}{-}}\NormalTok{ area }\SpecialCharTok{*} \FunctionTok{exp}\NormalTok{( distance }\SpecialCharTok{+}\NormalTok{ elevation }\SpecialCharTok{+}\NormalTok{ space }\SpecialCharTok{+}\NormalTok{ Intercept)}
\NormalTok{    expect }\OtherTok{\textless{}{-}}\NormalTok{ (}\DecValTok{1}\SpecialCharTok{{-}}\NormalTok{pi) }\SpecialCharTok{*}\NormalTok{ lambda}
\NormalTok{    variance }\OtherTok{\textless{}{-}}\NormalTok{ (}\DecValTok{1}\SpecialCharTok{{-}}\NormalTok{pi) }\SpecialCharTok{*}\NormalTok{ (lambda }\SpecialCharTok{+}\NormalTok{ pi }\SpecialCharTok{*}\NormalTok{ lambda}\SpecialCharTok{\^{}}\DecValTok{2}\NormalTok{)}
    \FunctionTok{list}\NormalTok{(}
      \AttributeTok{lambda =}\NormalTok{ lambda,}
      \AttributeTok{expect =}\NormalTok{ expect,}
      \AttributeTok{variance =}\NormalTok{ variance}
\NormalTok{    )}
\NormalTok{  \},}
  \AttributeTok{n.samples =} \DecValTok{2500}
\NormalTok{)}
\end{Highlighting}
\end{Shaded}

\end{tcolorbox}

\begin{figure}[H]

{\centering \includegraphics[width=0.8\linewidth,height=\textheight,keepaspectratio]{day5_practical_8_files/figure-pdf/unnamed-chunk-13-1.png}

}

\caption{Estimated \(\lambda\) (left) and expected counts (right) with
zero inflated model}

\end{figure}%

\subsection{Hurdle model (Type0)}\label{sec-zap}

We now fit a hurdle model to the same data.

In the \texttt{zeroinflatedpoisson0} model is defined by the
\href{https://inla.r-inla-download.org/r-inla.org/doc/likelihood/zeroinflated.pdf}{following
observation probability model}

\[
\text{Prob}(y\vert\dots)=\pi\times 1_{y=0}+(1-\pi)\times \text{Poisson}(y\vert y>0)
\]

where \(\pi\) is the probability of zero.

The expectation and variance of the counts are as follows:

\begin{equation}\phantomsection\label{eq-mean-zap}{
\begin{aligned}
E(\text{count})&=\frac{1}{1-\exp(-\lambda)}\pi\lambda \\
Var(\text{count})&=  E(\text{count}) \left(1-\exp(-\lambda) E(\text{count})\right)
\end{aligned}
}\end{equation}

\begin{tcolorbox}[enhanced jigsaw, colframe=quarto-callout-warning-color-frame, colback=white, colbacktitle=quarto-callout-warning-color!10!white, left=2mm, breakable, toptitle=1mm, toprule=.15mm, coltitle=black, bottomrule=.15mm, titlerule=0mm, leftrule=.75mm, rightrule=.15mm, title={Task}, bottomtitle=1mm, arc=.35mm, opacityback=0, opacitybacktitle=0.6]

Fit a hurdle model to the data using the \texttt{zeroinflatedpoisson0}
likelihood

Take hint

You do not need to redefine the components as the linear predictor is
not changing.

Click here to see the solution

\begin{Shaded}
\begin{Highlighting}[]
\NormalTok{lik }\OtherTok{=} \FunctionTok{bru\_obs}\NormalTok{(}\AttributeTok{formula =}\NormalTok{ count }\SpecialCharTok{\textasciitilde{}}\NormalTok{ .,}
    \AttributeTok{family =} \StringTok{"zeroinflatedpoisson0"}\NormalTok{, }
    \AttributeTok{data =}\NormalTok{ counts\_df,}
    \AttributeTok{E =}\NormalTok{ area)}

\NormalTok{fit\_zap }\OtherTok{\textless{}{-}} \FunctionTok{bru}\NormalTok{(cmp, lik)}
\end{Highlighting}
\end{Shaded}

\end{tcolorbox}

\begin{tcolorbox}[enhanced jigsaw, colframe=quarto-callout-warning-color-frame, colback=white, colbacktitle=quarto-callout-warning-color!10!white, left=2mm, breakable, toptitle=1mm, toprule=.15mm, coltitle=black, bottomrule=.15mm, titlerule=0mm, leftrule=.75mm, rightrule=.15mm, title={Task}, bottomtitle=1mm, arc=.35mm, opacityback=0, opacitybacktitle=0.6]

As before, check what the estimated probability of zero is and use
\texttt{predict()} to obtain a map of the estimated mean counts in
Equation~\ref{eq-mean-zap} over the domain.

Take hint

Click here to see the solution

\begin{Shaded}
\begin{Highlighting}[]

\NormalTok{pred\_zap }\OtherTok{\textless{}{-}} \FunctionTok{predict}\NormalTok{( fit\_zap, counts\_df,}
  \SpecialCharTok{\textasciitilde{}}\NormalTok{ \{}
\NormalTok{    pi }\OtherTok{\textless{}{-}}\NormalTok{ zero\_probability\_parameter\_for\_zero\_inflated\_poisson\_0}
\NormalTok{    lambda }\OtherTok{\textless{}{-}}\NormalTok{ area }\SpecialCharTok{*} \FunctionTok{exp}\NormalTok{( distance }\SpecialCharTok{+}\NormalTok{ elevation }\SpecialCharTok{+}\NormalTok{ space }\SpecialCharTok{+}\NormalTok{ Intercept)}
\NormalTok{    expect }\OtherTok{\textless{}{-}}\NormalTok{ ((}\DecValTok{1}\SpecialCharTok{{-}}\FunctionTok{exp}\NormalTok{(}\SpecialCharTok{{-}}\NormalTok{lambda))}\SpecialCharTok{\^{}}\NormalTok{(}\SpecialCharTok{{-}}\DecValTok{1}\NormalTok{) }\SpecialCharTok{*}\NormalTok{ pi }\SpecialCharTok{*}\NormalTok{ lambda)}
    \FunctionTok{list}\NormalTok{(}
      \AttributeTok{lambda =}\NormalTok{ lambda,}
      \AttributeTok{expect =}\NormalTok{ expect}
\NormalTok{    )}
\NormalTok{  \},}
  \AttributeTok{n.samples =} \DecValTok{2500}
\NormalTok{)}
\end{Highlighting}
\end{Shaded}

\end{tcolorbox}

\begin{figure}[H]

{\centering \includegraphics[width=0.8\linewidth,height=\textheight,keepaspectratio]{day5_practical_8_files/figure-pdf/unnamed-chunk-16-1.png}

}

\caption{Estimated \(\lambda\) (left) and expected counts (right) with
hurdle model}

\end{figure}%

\subsection{Hurdle model using two likelihoods}\label{sec-two-lik}

Here the model is the same as in Section~\ref{sec-zap}, but this time we
also want to model \(\pi\) using covariates and random effects.
Therefore we define a second linear predictor \[
\eta^2 =\beta_0^2 + \beta_1^2\text{Elevation} +  \beta_2^2\text{Distance} + u^2 
\] \textbf{Note} here we have defined the two linear predictor to use
the same covariates, but this is not necessary, they can be totally
independent.

To fit this model we have to define two likelihoods: - One will account
for the presence-absence process and has a Binomial model - One will
account for the counts and has a truncated Poisson model

\begin{tcolorbox}[enhanced jigsaw, colframe=quarto-callout-warning-color-frame, colback=white, colbacktitle=quarto-callout-warning-color!10!white, left=2mm, breakable, toptitle=1mm, toprule=.15mm, coltitle=black, bottomrule=.15mm, titlerule=0mm, leftrule=.75mm, rightrule=.15mm, title={Task}, bottomtitle=1mm, arc=.35mm, opacityback=0, opacitybacktitle=0.6]

Complete the following code to fit a hurdle model based on two
likelihoods:

\begin{Shaded}
\begin{Highlighting}[]
\CommentTok{\# define components}
\NormalTok{cmp }\OtherTok{\textless{}{-}} \ErrorTok{\textasciitilde{}}
  \FunctionTok{Intercept\_count}\NormalTok{(}\DecValTok{1}\NormalTok{) }\SpecialCharTok{+}
    \FunctionTok{elev\_count}\NormalTok{(elev\_cov1, }\AttributeTok{model =} \StringTok{"linear"}\NormalTok{) }\SpecialCharTok{+}
    \FunctionTok{dist\_count}\NormalTok{(dist\_cov1, }\AttributeTok{model =} \StringTok{"linear"}\NormalTok{) }\SpecialCharTok{+}
    \FunctionTok{space\_count}\NormalTok{(geometry, }\AttributeTok{model =}\NormalTok{ matern) }\SpecialCharTok{+}
    \FunctionTok{Intercept\_presence}\NormalTok{(}\DecValTok{1}\NormalTok{) }\SpecialCharTok{+}
    \FunctionTok{elev\_presence}\NormalTok{(elev\_cov1, }\AttributeTok{model =} \StringTok{"linear"}\NormalTok{) }\SpecialCharTok{+}
    \FunctionTok{dist\_presence}\NormalTok{(dist\_cov1, }\AttributeTok{model =} \StringTok{"linear"}\NormalTok{) }\SpecialCharTok{+}
    \FunctionTok{space\_presence}\NormalTok{(geometry, }\AttributeTok{model =}\NormalTok{ matern) }

\CommentTok{\# positive count model}
\NormalTok{pos\_count\_obs }\OtherTok{\textless{}{-}} \FunctionTok{bru\_obs}\NormalTok{(}\AttributeTok{formula =}\NormalTok{ ...,}
      \AttributeTok{family =}\NormalTok{ ...,}
      \AttributeTok{data =}\NormalTok{ counts\_df[counts\_df}\SpecialCharTok{$}\NormalTok{present }\SpecialCharTok{\textgreater{}} \DecValTok{0}\NormalTok{, ],}
      \AttributeTok{E =}\NormalTok{ area)}
  
\CommentTok{\# presence model}
\NormalTok{presence\_obs }\OtherTok{\textless{}{-}} \FunctionTok{bru\_obs}\NormalTok{(formula ...,}
  \AttributeTok{family =}\NormalTok{ ...,}
  \AttributeTok{data =}\NormalTok{ counts\_df,}
\NormalTok{)}

\CommentTok{\# fit the model}
\NormalTok{fit\_zap2 }\OtherTok{\textless{}{-}} \FunctionTok{bru}\NormalTok{(...)}
\end{Highlighting}
\end{Shaded}

Take hint

Add hint details here\ldots{}

Click here to see the solution

\begin{Shaded}
\begin{Highlighting}[]
\NormalTok{cmp }\OtherTok{\textless{}{-}} \ErrorTok{\textasciitilde{}}
  \FunctionTok{Intercept\_count}\NormalTok{(}\DecValTok{1}\NormalTok{) }\SpecialCharTok{+}
    \FunctionTok{elev\_count}\NormalTok{(elev\_cov1, }\AttributeTok{model =} \StringTok{"linear"}\NormalTok{) }\SpecialCharTok{+}
    \FunctionTok{dist\_count}\NormalTok{(dist\_cov1, }\AttributeTok{model =} \StringTok{"linear"}\NormalTok{) }\SpecialCharTok{+}
    \FunctionTok{space\_count}\NormalTok{(geometry, }\AttributeTok{model =}\NormalTok{ matern) }\SpecialCharTok{+}
    \FunctionTok{Intercept\_presence}\NormalTok{(}\DecValTok{1}\NormalTok{) }\SpecialCharTok{+}
    \FunctionTok{elev\_presence}\NormalTok{(elev\_cov1, }\AttributeTok{model =} \StringTok{"linear"}\NormalTok{) }\SpecialCharTok{+}
    \FunctionTok{dist\_presence}\NormalTok{(dist\_cov1, }\AttributeTok{model =} \StringTok{"linear"}\NormalTok{) }\SpecialCharTok{+}
    \FunctionTok{space\_presence}\NormalTok{(geometry, }\AttributeTok{model =}\NormalTok{ matern) }


\NormalTok{pos\_count\_obs }\OtherTok{\textless{}{-}} \FunctionTok{bru\_obs}\NormalTok{(}\AttributeTok{formula =}\NormalTok{ count }\SpecialCharTok{\textasciitilde{}}\NormalTok{ Intercept\_count }\SpecialCharTok{+}\NormalTok{ elev\_count }\SpecialCharTok{+} 
\NormalTok{                                   dist\_count }\SpecialCharTok{+}\NormalTok{ space\_count,}
      \AttributeTok{family =} \StringTok{"nzpoisson"}\NormalTok{,}
      \AttributeTok{data =}\NormalTok{ counts\_df[counts\_df}\SpecialCharTok{$}\NormalTok{present }\SpecialCharTok{\textgreater{}} \DecValTok{0}\NormalTok{, ],}
      \AttributeTok{E =}\NormalTok{ area)}
  

\NormalTok{presence\_obs }\OtherTok{\textless{}{-}} \FunctionTok{bru\_obs}\NormalTok{(}\AttributeTok{formula =}\NormalTok{ present }\SpecialCharTok{\textasciitilde{}}\NormalTok{ Intercept\_presence }\SpecialCharTok{+}\NormalTok{ elev\_presence }\SpecialCharTok{+}\NormalTok{ dist\_presence }\SpecialCharTok{+}
\NormalTok{                          space\_presence,}
  \AttributeTok{family =} \StringTok{"binomial"}\NormalTok{,}
  \AttributeTok{data =}\NormalTok{ counts\_df,}
\NormalTok{)}

\NormalTok{fit\_zap2 }\OtherTok{\textless{}{-}} \FunctionTok{bru}\NormalTok{(}
\NormalTok{  cmp,}
\NormalTok{  presence\_obs,}
\NormalTok{  pos\_count\_obs}
\NormalTok{)}
\end{Highlighting}
\end{Shaded}

\end{tcolorbox}

\subsection{Hurdle model using two likelihoods and a shared
component}\label{sec-two-lik-share}

Note that in the model above, there is no direct link between the
parameters of the two observation parts, and we could estimate them
separately. However, the two likelihoods could share some of the
components; for example the \texttt{space\_count} component could be
used for both predictors. This would be possible using the \texttt{copy}
argument.

We would then need to define one component as
\texttt{space(geometry,\ model\ =\ matern)} and then a copy of it as
\texttt{space\_copy(geometry,\ copy\ =\ "space",\ fixed\ =\ FALSE)}.

The results from the model in \textbf{?@sec-sec-two-lik} show that the
estimated covariance parameters for the two fields are very different,
so it is probably not sensible to share the same component between the
two parts. We do it anyway to show an example:

\begin{Shaded}
\begin{Highlighting}[]
\NormalTok{cmp }\OtherTok{\textless{}{-}} \ErrorTok{\textasciitilde{}}
  \FunctionTok{Intercept\_count}\NormalTok{(}\DecValTok{1}\NormalTok{) }\SpecialCharTok{+}
    \FunctionTok{elev\_count}\NormalTok{(elev\_cov1, }\AttributeTok{model =} \StringTok{"linear"}\NormalTok{) }\SpecialCharTok{+}
    \FunctionTok{dist\_count}\NormalTok{(dist\_cov1, }\AttributeTok{model =} \StringTok{"linear"}\NormalTok{) }\SpecialCharTok{+}
    \FunctionTok{Intercept\_presence}\NormalTok{(}\DecValTok{1}\NormalTok{) }\SpecialCharTok{+}
    \FunctionTok{elev\_presence}\NormalTok{(elev\_cov1, }\AttributeTok{model =} \StringTok{"linear"}\NormalTok{) }\SpecialCharTok{+}
    \FunctionTok{dist\_presence}\NormalTok{(dist\_cov1, }\AttributeTok{model =} \StringTok{"linear"}\NormalTok{) }\SpecialCharTok{+}
    \FunctionTok{space}\NormalTok{(geometry, }\AttributeTok{model =}\NormalTok{ matern) }\SpecialCharTok{+}
  \FunctionTok{space\_copy}\NormalTok{(geometry, }\AttributeTok{copy =} \StringTok{"space"}\NormalTok{, }\AttributeTok{fixed =} \ConstantTok{FALSE}\NormalTok{)}


\NormalTok{pos\_count\_obs }\OtherTok{\textless{}{-}} \FunctionTok{bru\_obs}\NormalTok{(}\AttributeTok{formula =}\NormalTok{ count }\SpecialCharTok{\textasciitilde{}}\NormalTok{ Intercept\_count }\SpecialCharTok{+}\NormalTok{ elev\_count }\SpecialCharTok{+}\NormalTok{ dist\_count }\SpecialCharTok{+}\NormalTok{ space,}
      \AttributeTok{family =} \StringTok{"nzpoisson"}\NormalTok{,}
      \AttributeTok{data =}\NormalTok{ counts\_df[counts\_df}\SpecialCharTok{$}\NormalTok{present }\SpecialCharTok{\textgreater{}} \DecValTok{0}\NormalTok{, ],}
      \AttributeTok{E =}\NormalTok{ area)}

\NormalTok{presence\_obs }\OtherTok{\textless{}{-}} \FunctionTok{bru\_obs}\NormalTok{(}\AttributeTok{formula =}\NormalTok{ present }\SpecialCharTok{\textasciitilde{}}\NormalTok{ Intercept\_presence }\SpecialCharTok{+}\NormalTok{ elev\_presence }\SpecialCharTok{+}\NormalTok{ dist\_presence }\SpecialCharTok{+}\NormalTok{ space\_copy,}
  \AttributeTok{family =} \StringTok{"binomial"}\NormalTok{,}
  \AttributeTok{data =}\NormalTok{ counts\_df)}

\NormalTok{fit\_zap3 }\OtherTok{\textless{}{-}} \FunctionTok{bru}\NormalTok{(}
\NormalTok{  cmp,}
\NormalTok{  presence\_obs,}
\NormalTok{  pos\_count\_obs)}
\end{Highlighting}
\end{Shaded}

\subsection{Comparing models}\label{comparing-models}

We have fitted four different models. Now we want to compare them and
see how they fit the data.

\subsubsection{Comparing model
predictions}\label{comparing-model-predictions}

We first want to compare the estimated surfaces of expected counts. To
do this we want to produce the estimated expected counts, similar to
what we did in Section~\ref{sec-zip} and Section~\ref{sec-zap} for all
four models and plot them together:

\begin{Shaded}
\begin{Highlighting}[]
\NormalTok{pred\_zip }\OtherTok{\textless{}{-}} \FunctionTok{predict}\NormalTok{(}
\NormalTok{  fit\_zip, }
\NormalTok{  counts\_df,}
  \SpecialCharTok{\textasciitilde{}}\NormalTok{ \{}
\NormalTok{    pi }\OtherTok{\textless{}{-}}\NormalTok{ zero\_probability\_parameter\_for\_zero\_inflated\_poisson\_1}
\NormalTok{    lambda }\OtherTok{\textless{}{-}}\NormalTok{ area }\SpecialCharTok{*} \FunctionTok{exp}\NormalTok{( distance }\SpecialCharTok{+}\NormalTok{ elevation }\SpecialCharTok{+}\NormalTok{ space }\SpecialCharTok{+}\NormalTok{ Intercept)}
\NormalTok{    expect }\OtherTok{\textless{}{-}}\NormalTok{ (}\DecValTok{1}\SpecialCharTok{{-}}\NormalTok{pi) }\SpecialCharTok{*}\NormalTok{ lambda}
\NormalTok{    variance }\OtherTok{\textless{}{-}}\NormalTok{ (}\DecValTok{1}\SpecialCharTok{{-}}\NormalTok{pi) }\SpecialCharTok{*}\NormalTok{ (lambda }\SpecialCharTok{+}\NormalTok{ pi }\SpecialCharTok{*}\NormalTok{ lambda}\SpecialCharTok{\^{}}\DecValTok{2}\NormalTok{)}
    \FunctionTok{list}\NormalTok{(}
      \AttributeTok{expect =}\NormalTok{ expect}
\NormalTok{    )}
\NormalTok{  \},}\AttributeTok{n.samples =} \DecValTok{2500}\NormalTok{)}

\NormalTok{pred\_zap }\OtherTok{\textless{}{-}} \FunctionTok{predict}\NormalTok{( fit\_zap, counts\_df,}
  \SpecialCharTok{\textasciitilde{}}\NormalTok{ \{}
\NormalTok{    pi }\OtherTok{\textless{}{-}}\NormalTok{ zero\_probability\_parameter\_for\_zero\_inflated\_poisson\_0}
\NormalTok{    lambda }\OtherTok{\textless{}{-}}\NormalTok{ area }\SpecialCharTok{*} \FunctionTok{exp}\NormalTok{( distance }\SpecialCharTok{+}\NormalTok{ elevation }\SpecialCharTok{+}\NormalTok{ space }\SpecialCharTok{+}\NormalTok{ Intercept)}
\NormalTok{    expect }\OtherTok{\textless{}{-}}\NormalTok{ ((}\DecValTok{1}\SpecialCharTok{{-}}\FunctionTok{exp}\NormalTok{(}\SpecialCharTok{{-}}\NormalTok{lambda))}\SpecialCharTok{\^{}}\NormalTok{(}\SpecialCharTok{{-}}\DecValTok{1}\NormalTok{) }\SpecialCharTok{*}\NormalTok{ pi }\SpecialCharTok{*}\NormalTok{ lambda)}
    \FunctionTok{list}\NormalTok{(}
      \AttributeTok{expect =}\NormalTok{ expect)}
\NormalTok{  \},}\AttributeTok{n.samples =} \DecValTok{2500}\NormalTok{)}

\NormalTok{inv.logit }\OtherTok{=} \ControlFlowTok{function}\NormalTok{(x) (}\FunctionTok{exp}\NormalTok{(x)}\SpecialCharTok{/}\NormalTok{(}\DecValTok{1}\SpecialCharTok{+}\FunctionTok{exp}\NormalTok{(x)))}

\NormalTok{pred\_zap2 }\OtherTok{\textless{}{-}} \FunctionTok{predict}\NormalTok{( fit\_zap2, counts\_df,}
  \SpecialCharTok{\textasciitilde{}}\NormalTok{ \{}
\NormalTok{    pi }\OtherTok{\textless{}{-}} \FunctionTok{inv.logit}\NormalTok{(Intercept\_presence }\SpecialCharTok{+}\NormalTok{ elev\_presence }\SpecialCharTok{+}\NormalTok{ dist\_presence }\SpecialCharTok{+}\NormalTok{ space\_presence)}
\NormalTok{    lambda }\OtherTok{\textless{}{-}}\NormalTok{ area }\SpecialCharTok{*} \FunctionTok{exp}\NormalTok{( dist\_count }\SpecialCharTok{+}\NormalTok{ elev\_count }\SpecialCharTok{+}\NormalTok{ space\_count }\SpecialCharTok{+}\NormalTok{ Intercept\_count)}
\NormalTok{    expect }\OtherTok{\textless{}{-}}\NormalTok{ ((}\DecValTok{1}\SpecialCharTok{{-}}\FunctionTok{exp}\NormalTok{(}\SpecialCharTok{{-}}\NormalTok{lambda))}\SpecialCharTok{\^{}}\NormalTok{(}\SpecialCharTok{{-}}\DecValTok{1}\NormalTok{) }\SpecialCharTok{*}\NormalTok{ pi }\SpecialCharTok{*}\NormalTok{ lambda)}
    \FunctionTok{list}\NormalTok{(}
      \AttributeTok{expect =}\NormalTok{ expect)}
\NormalTok{  \},}\AttributeTok{n.samples =} \DecValTok{2500}\NormalTok{)}

\NormalTok{pred\_zap3 }\OtherTok{\textless{}{-}} \FunctionTok{predict}\NormalTok{( fit\_zap3, counts\_df,}
  \SpecialCharTok{\textasciitilde{}}\NormalTok{ \{}
\NormalTok{    pi }\OtherTok{\textless{}{-}} \FunctionTok{inv.logit}\NormalTok{(Intercept\_presence }\SpecialCharTok{+}\NormalTok{ elev\_presence }\SpecialCharTok{+}\NormalTok{ dist\_presence }\SpecialCharTok{+}\NormalTok{ space\_copy)}
\NormalTok{    lambda }\OtherTok{\textless{}{-}}\NormalTok{ area }\SpecialCharTok{*} \FunctionTok{exp}\NormalTok{( dist\_count }\SpecialCharTok{+}\NormalTok{ elev\_count }\SpecialCharTok{+}\NormalTok{ space }\SpecialCharTok{+}\NormalTok{ Intercept\_count)}
\NormalTok{    expect }\OtherTok{\textless{}{-}}\NormalTok{ ((}\DecValTok{1}\SpecialCharTok{{-}}\FunctionTok{exp}\NormalTok{(}\SpecialCharTok{{-}}\NormalTok{lambda))}\SpecialCharTok{\^{}}\NormalTok{(}\SpecialCharTok{{-}}\DecValTok{1}\NormalTok{) }\SpecialCharTok{*}\NormalTok{ pi }\SpecialCharTok{*}\NormalTok{ lambda)}
    \FunctionTok{list}\NormalTok{(}
      \AttributeTok{expect =}\NormalTok{ expect)}
\NormalTok{  \},}\AttributeTok{n.samples =} \DecValTok{2500}\NormalTok{)}




  \FunctionTok{data.frame}\NormalTok{(}\AttributeTok{x =} \FunctionTok{st\_coordinates}\NormalTok{(counts\_df)[,}\DecValTok{1}\NormalTok{],}
             \AttributeTok{y =} \FunctionTok{st\_coordinates}\NormalTok{(counts\_df)[,}\DecValTok{2}\NormalTok{],}
    \AttributeTok{zip =}\NormalTok{ pred\_zip}\SpecialCharTok{$}\NormalTok{expect}\SpecialCharTok{$}\NormalTok{mean,}
         \AttributeTok{hurdle =}\NormalTok{ pred\_zap}\SpecialCharTok{$}\NormalTok{expect}\SpecialCharTok{$}\NormalTok{mean,}
         \AttributeTok{hurdle2 =}\NormalTok{ pred\_zap2}\SpecialCharTok{$}\NormalTok{expect}\SpecialCharTok{$}\NormalTok{mean,}
         \AttributeTok{hurdle3 =}\NormalTok{ pred\_zap3}\SpecialCharTok{$}\NormalTok{expect}\SpecialCharTok{$}\NormalTok{mean)  }\SpecialCharTok{\%\textgreater{}\%}
  \FunctionTok{pivot\_longer}\NormalTok{(}\SpecialCharTok{{-}}\FunctionTok{c}\NormalTok{(x,y)) }\SpecialCharTok{\%\textgreater{}\%}
  \FunctionTok{ggplot}\NormalTok{() }\SpecialCharTok{+} \FunctionTok{geom\_tile}\NormalTok{(}\FunctionTok{aes}\NormalTok{(x,y, }\AttributeTok{fill =}\NormalTok{ value)) }\SpecialCharTok{+} \FunctionTok{facet\_wrap}\NormalTok{(.}\SpecialCharTok{\textasciitilde{}}\NormalTok{name) }\SpecialCharTok{+}
\NormalTok{    theme\_map }\SpecialCharTok{+} \FunctionTok{scale\_fill\_scico}\NormalTok{(}\AttributeTok{direction =} \SpecialCharTok{{-}}\DecValTok{1}\NormalTok{)}
\end{Highlighting}
\end{Shaded}

\begin{figure}[H]

{\centering \pandocbounded{\includegraphics[keepaspectratio]{day5_practical_8_files/figure-pdf/unnamed-chunk-19-1.png}}

}

\caption{Estimated expected counts for all four models}

\end{figure}%

\begin{tcolorbox}[enhanced jigsaw, colframe=quarto-callout-warning-color-frame, colback=white, colbacktitle=quarto-callout-warning-color!10!white, left=2mm, breakable, toptitle=1mm, toprule=.15mm, coltitle=black, bottomrule=.15mm, titlerule=0mm, leftrule=.75mm, rightrule=.15mm, title={Task}, bottomtitle=1mm, arc=.35mm, opacityback=0, opacitybacktitle=0.6]

Create plots of the estimated variance of the counts.

Take hint

The fomulas for the variances are in Equation~\ref{eq-mean-zip} and
Equation~\ref{eq-mean-zap}.

Click here to see the solution

\begin{Shaded}
\begin{Highlighting}[]
\NormalTok{pred\_zip }\OtherTok{\textless{}{-}} \FunctionTok{predict}\NormalTok{(}
\NormalTok{  fit\_zip, }
\NormalTok{  counts\_df,}
  \SpecialCharTok{\textasciitilde{}}\NormalTok{ \{}
\NormalTok{    pi }\OtherTok{\textless{}{-}}\NormalTok{ zero\_probability\_parameter\_for\_zero\_inflated\_poisson\_1}
\NormalTok{    lambda }\OtherTok{\textless{}{-}}\NormalTok{ area }\SpecialCharTok{*} \FunctionTok{exp}\NormalTok{( distance }\SpecialCharTok{+}\NormalTok{ elevation }\SpecialCharTok{+}\NormalTok{ space }\SpecialCharTok{+}\NormalTok{ Intercept)}
\NormalTok{    variance }\OtherTok{\textless{}{-}}\NormalTok{ (}\DecValTok{1}\SpecialCharTok{{-}}\NormalTok{pi) }\SpecialCharTok{*}\NormalTok{ (lambda }\SpecialCharTok{+}\NormalTok{ pi }\SpecialCharTok{*}\NormalTok{ lambda}\SpecialCharTok{\^{}}\DecValTok{2}\NormalTok{)}
    \FunctionTok{list}\NormalTok{( }\AttributeTok{variance =}\NormalTok{ variance)}
\NormalTok{  \},}\AttributeTok{n.samples =} \DecValTok{2500}\NormalTok{)}

\NormalTok{pred\_zap }\OtherTok{\textless{}{-}} \FunctionTok{predict}\NormalTok{( fit\_zap, counts\_df,}
  \SpecialCharTok{\textasciitilde{}}\NormalTok{ \{}
\NormalTok{    pi }\OtherTok{\textless{}{-}}\NormalTok{ zero\_probability\_parameter\_for\_zero\_inflated\_poisson\_0}
\NormalTok{    lambda }\OtherTok{\textless{}{-}}\NormalTok{ area }\SpecialCharTok{*} \FunctionTok{exp}\NormalTok{( distance }\SpecialCharTok{+}\NormalTok{ elevation }\SpecialCharTok{+}\NormalTok{ space }\SpecialCharTok{+}\NormalTok{ Intercept)}
\NormalTok{    expect }\OtherTok{\textless{}{-}}\NormalTok{ ((}\DecValTok{1}\SpecialCharTok{{-}}\FunctionTok{exp}\NormalTok{(}\SpecialCharTok{{-}}\NormalTok{lambda))}\SpecialCharTok{\^{}}\NormalTok{(}\SpecialCharTok{{-}}\DecValTok{1}\NormalTok{) }\SpecialCharTok{*}\NormalTok{ pi }\SpecialCharTok{*}\NormalTok{ lambda)}
\NormalTok{    variance }\OtherTok{=}\NormalTok{ expect }\SpecialCharTok{*}\NormalTok{(}\DecValTok{1}\SpecialCharTok{{-}}\FunctionTok{exp}\NormalTok{(}\SpecialCharTok{{-}}\NormalTok{lambda) }\SpecialCharTok{*}\NormalTok{ expect)}
    \FunctionTok{list}\NormalTok{(}\AttributeTok{variance =}\NormalTok{ variance)}
\NormalTok{  \},}
  \AttributeTok{n.samples =} \DecValTok{2500}\NormalTok{)}

\NormalTok{inv.logit }\OtherTok{=} \ControlFlowTok{function}\NormalTok{(x) (}\FunctionTok{exp}\NormalTok{(x)}\SpecialCharTok{/}\NormalTok{(}\DecValTok{1}\SpecialCharTok{+}\FunctionTok{exp}\NormalTok{(x)))}

\NormalTok{pred\_zap2 }\OtherTok{\textless{}{-}} \FunctionTok{predict}\NormalTok{( fit\_zap2, counts\_df,}
  \SpecialCharTok{\textasciitilde{}}\NormalTok{ \{}
\NormalTok{    pi }\OtherTok{\textless{}{-}} \FunctionTok{inv.logit}\NormalTok{(Intercept\_presence }\SpecialCharTok{+}\NormalTok{ elev\_presence }\SpecialCharTok{+}\NormalTok{ dist\_presence }\SpecialCharTok{+}\NormalTok{ space\_presence)}
\NormalTok{    lambda }\OtherTok{\textless{}{-}}\NormalTok{ area }\SpecialCharTok{*} \FunctionTok{exp}\NormalTok{( dist\_count }\SpecialCharTok{+}\NormalTok{ elev\_count }\SpecialCharTok{+}\NormalTok{ space\_count }\SpecialCharTok{+}\NormalTok{ Intercept\_count)}
\NormalTok{    expect }\OtherTok{\textless{}{-}}\NormalTok{ ((}\DecValTok{1}\SpecialCharTok{{-}}\FunctionTok{exp}\NormalTok{(}\SpecialCharTok{{-}}\NormalTok{lambda))}\SpecialCharTok{\^{}}\NormalTok{(}\SpecialCharTok{{-}}\DecValTok{1}\NormalTok{) }\SpecialCharTok{*}\NormalTok{ pi }\SpecialCharTok{*}\NormalTok{ lambda)}
\NormalTok{    variance }\OtherTok{=}\NormalTok{ expect }\SpecialCharTok{*}\NormalTok{(}\DecValTok{1}\SpecialCharTok{{-}}\FunctionTok{exp}\NormalTok{(}\SpecialCharTok{{-}}\NormalTok{lambda) }\SpecialCharTok{*}\NormalTok{ expect)}
    \FunctionTok{list}\NormalTok{(}\AttributeTok{variance =}\NormalTok{ variance)}
\NormalTok{  \},}
  \AttributeTok{n.samples =} \DecValTok{2500}\NormalTok{)}

\NormalTok{pred\_zap3 }\OtherTok{\textless{}{-}} \FunctionTok{predict}\NormalTok{( fit\_zap3, counts\_df,}
  \SpecialCharTok{\textasciitilde{}}\NormalTok{ \{}
\NormalTok{    pi }\OtherTok{\textless{}{-}} \FunctionTok{inv.logit}\NormalTok{(Intercept\_presence }\SpecialCharTok{+}\NormalTok{ elev\_presence }\SpecialCharTok{+}\NormalTok{ dist\_presence }\SpecialCharTok{+}\NormalTok{ space\_copy)}
\NormalTok{    lambda }\OtherTok{\textless{}{-}}\NormalTok{ area }\SpecialCharTok{*} \FunctionTok{exp}\NormalTok{( dist\_count }\SpecialCharTok{+}\NormalTok{ elev\_count }\SpecialCharTok{+}\NormalTok{ space }\SpecialCharTok{+}\NormalTok{ Intercept\_count)}
\NormalTok{    expect }\OtherTok{\textless{}{-}}\NormalTok{ ((}\DecValTok{1}\SpecialCharTok{{-}}\FunctionTok{exp}\NormalTok{(}\SpecialCharTok{{-}}\NormalTok{lambda))}\SpecialCharTok{\^{}}\NormalTok{(}\SpecialCharTok{{-}}\DecValTok{1}\NormalTok{) }\SpecialCharTok{*}\NormalTok{ pi }\SpecialCharTok{*}\NormalTok{ lambda)}
\NormalTok{    variance }\OtherTok{=}\NormalTok{ expect }\SpecialCharTok{*}\NormalTok{(}\DecValTok{1}\SpecialCharTok{{-}}\FunctionTok{exp}\NormalTok{(}\SpecialCharTok{{-}}\NormalTok{lambda) }\SpecialCharTok{*}\NormalTok{ expect)}
    \FunctionTok{list}\NormalTok{(}\AttributeTok{variance =}\NormalTok{ variance)}
\NormalTok{  \},}
  \AttributeTok{n.samples =} \DecValTok{2500}\NormalTok{)}




  \FunctionTok{data.frame}\NormalTok{(}\AttributeTok{x =} \FunctionTok{st\_coordinates}\NormalTok{(counts\_df)[,}\DecValTok{1}\NormalTok{],}
             \AttributeTok{y =} \FunctionTok{st\_coordinates}\NormalTok{(counts\_df)[,}\DecValTok{2}\NormalTok{],}
    \AttributeTok{zip =}\NormalTok{ pred\_zip}\SpecialCharTok{$}\NormalTok{variance}\SpecialCharTok{$}\NormalTok{mean,}
         \AttributeTok{hurdle =}\NormalTok{ pred\_zap}\SpecialCharTok{$}\NormalTok{variance}\SpecialCharTok{$}\NormalTok{mean,}
         \AttributeTok{hurdle2 =}\NormalTok{ pred\_zap2}\SpecialCharTok{$}\NormalTok{variance}\SpecialCharTok{$}\NormalTok{mean,}
         \AttributeTok{hurdle3 =}\NormalTok{ pred\_zap3}\SpecialCharTok{$}\NormalTok{variance}\SpecialCharTok{$}\NormalTok{mean)  }\SpecialCharTok{\%\textgreater{}\%}
  \FunctionTok{pivot\_longer}\NormalTok{(}\SpecialCharTok{{-}}\FunctionTok{c}\NormalTok{(x,y)) }\SpecialCharTok{\%\textgreater{}\%}
  \FunctionTok{ggplot}\NormalTok{() }\SpecialCharTok{+} \FunctionTok{geom\_tile}\NormalTok{(}\FunctionTok{aes}\NormalTok{(x,y, }\AttributeTok{fill =}\NormalTok{ value)) }\SpecialCharTok{+} \FunctionTok{facet\_wrap}\NormalTok{(.}\SpecialCharTok{\textasciitilde{}}\NormalTok{name) }\SpecialCharTok{+}
\NormalTok{    theme\_map }\SpecialCharTok{+} \FunctionTok{scale\_fill\_scico}\NormalTok{(}\AttributeTok{direction =} \SpecialCharTok{{-}}\DecValTok{1}\NormalTok{)}
\end{Highlighting}
\end{Shaded}

\begin{center}
\pandocbounded{\includegraphics[keepaspectratio]{day5_practical_8_files/figure-pdf/unnamed-chunk-20-1.png}}
\end{center}

\end{tcolorbox}

\subsubsection{Using scores}\label{using-scores}

We can compare model using the scores that the \emph{bru()} function
computes since we have set, at the beginning. the options to :::
\{.cell\}

\begin{Shaded}
\begin{Highlighting}[]
\FunctionTok{bru\_options\_set}\NormalTok{(}\AttributeTok{control.compute =} \FunctionTok{list}\NormalTok{(}\AttributeTok{dic =} \ConstantTok{TRUE}\NormalTok{,}
                                       \AttributeTok{waic =} \ConstantTok{TRUE}\NormalTok{,}
                                       \AttributeTok{mlik =} \ConstantTok{TRUE}\NormalTok{,}
                                       \AttributeTok{cpo =} \ConstantTok{TRUE}\NormalTok{))}
\end{Highlighting}
\end{Shaded}

:::

Lets use these scores to compare the models.

\begin{tcolorbox}[enhanced jigsaw, colframe=quarto-callout-warning-color-frame, colback=white, colbacktitle=quarto-callout-warning-color!10!white, left=2mm, breakable, toptitle=1mm, toprule=.15mm, coltitle=black, bottomrule=.15mm, titlerule=0mm, leftrule=.75mm, rightrule=.15mm, title={Task}, bottomtitle=1mm, arc=.35mm, opacityback=0, opacitybacktitle=0.6]

Extract the DIC, WAIC and MLIK values for the four models and compare
them

Click here to see the solution

\begin{Shaded}
\begin{Highlighting}[]
\FunctionTok{data.frame}\NormalTok{( }\AttributeTok{Model =} \FunctionTok{c}\NormalTok{(}\StringTok{"ZIP"}\NormalTok{, }\StringTok{"HURDLE"}\NormalTok{, }\StringTok{"HURDLE\_2"}\NormalTok{,}\StringTok{"HURDLE\_3"}\NormalTok{ ),}
  \AttributeTok{DIC =} \FunctionTok{c}\NormalTok{(fit\_zip}\SpecialCharTok{$}\NormalTok{dic}\SpecialCharTok{$}\NormalTok{dic, fit\_zap}\SpecialCharTok{$}\NormalTok{dic}\SpecialCharTok{$}\NormalTok{dic, }\AttributeTok{WAIC =}\NormalTok{ fit\_zap2}\SpecialCharTok{$}\NormalTok{dic}\SpecialCharTok{$}\NormalTok{dic, fit\_zap3}\SpecialCharTok{$}\NormalTok{dic}\SpecialCharTok{$}\NormalTok{dic),}
            \AttributeTok{WAIC =} \FunctionTok{c}\NormalTok{(fit\_zip}\SpecialCharTok{$}\NormalTok{waic}\SpecialCharTok{$}\NormalTok{waic, fit\_zap}\SpecialCharTok{$}\NormalTok{waic}\SpecialCharTok{$}\NormalTok{waic, fit\_zap2}\SpecialCharTok{$}\NormalTok{waic}\SpecialCharTok{$}\NormalTok{waic, fit\_zap3}\SpecialCharTok{$}\NormalTok{waic}\SpecialCharTok{$}\NormalTok{waic),}
            \AttributeTok{MLIK =} \FunctionTok{c}\NormalTok{(fit\_zip}\SpecialCharTok{$}\NormalTok{mlik[}\DecValTok{1}\NormalTok{], fit\_zap}\SpecialCharTok{$}\NormalTok{mlik[}\DecValTok{1}\NormalTok{], fit\_zap2}\SpecialCharTok{$}\NormalTok{mlik[}\DecValTok{1}\NormalTok{], fit\_zap3}\SpecialCharTok{$}\NormalTok{mlik[}\DecValTok{1}\NormalTok{]))}
\CommentTok{\#\textgreater{}      Model      DIC     WAIC      MLIK}
\CommentTok{\#\textgreater{} 1      ZIP 1214.131 1223.710 {-}686.9503}
\CommentTok{\#\textgreater{} 2   HURDLE 1886.066 1909.722 {-}994.3807}
\CommentTok{\#\textgreater{} 3 HURDLE\_2 1268.322 1285.524 {-}734.3697}
\CommentTok{\#\textgreater{} 4 HURDLE\_3 1635.274 2283.052 {-}858.2937}
\end{Highlighting}
\end{Shaded}

\end{tcolorbox}

From the table above we can see that the model that best balances
complexity and fit is the zero inflated one (ZIP).




\end{document}
